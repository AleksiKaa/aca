\documentclass[11pt,a4paper]{article}
\usepackage{hyperref}
\usepackage[left=1.5in,right=1.5in,top=1.5in,bottom=1.5in]{geometry}
\usepackage{enumitem}
\usepackage{amssymb,amsmath,mathrsfs,amsthm}
\usepackage{parskip, graphicx, float, wrapfig, subcaption}


\setlength{\parindent}{0pt}
\setlength{\parskip}{3mm}

\pagestyle{empty}

%Title and formatting stuff
\usepackage{titlesec}
\titleformat{\chapter}
{\normalfont\Huge\color{aaltoBlue}}{\thechapter}{20pt}{\filleft\Huge}[\vskip 0.5ex\titlerule]
\usepackage{fancyhdr}
\pagestyle{fancy}
\fancyhead[L]{}
\newcommand{\mymarginpar}[1]{\marginpar{\it \small \raggedleft #1}}
\renewcommand*\thesection{\arabic{section}}

% Miscellaneous LaTeX macros
\newcommand{\heading}[1]{{\Large\color{blue!80!black}\textbf{#1}}\par}
\newcommand{\slemp}[1]{{\color{red}#1}}
\newcommand{\slred}[1]{{\color{red}#1}}
\newcommand{\slbf}[1]{{\color{blue}\textbf{#1}}}
\newcommand{\pair}[2]{\langle {#1}, {#2} \rangle}
\newcommand{\Abar}{\bar{A}}
\newcommand{\ALG}{\text{ALG}}
\newcommand{\approxA}{\alpha_{\cal A}}
\newcommand{\argmin}{\operatornamewithlimits{arg\ min}}
\newcommand{\assign}{\mbox{$\; \leftarrow \;$}}
\newcommand{\bbar}{\bar{b}}
\newcommand{\Bin}{\text{Bin}}
\newcommand{\calA}{{\cal A}}
\newcommand{\calC}{{\cal C}}
\newcommand{\calI}{{\cal I}}
\newcommand{\calM}{{\cal M}}
\newcommand{\calN}{{\cal N}}
\newcommand{\calS}{{\cal S}}
\newcommand{\chat}{\hat{c}}
\newcommand{\chatopt}{\hat{c}^*}
\newcommand{\copt}{c^*}
\newcommand{\Cov}{\text{Cov}}
\newcommand{\DC}{\text{DC}}
\newcommand{\delC}{\partial C}
\newcommand{\delS}{\partial S}
\newcommand{\delT}{\partial T}
\newcommand{\diag}{\textit{diag}}
\newcommand{\dtv}{d_{\text{V}}}
\newcommand{\dom}{\mathcal{D}}
\newcommand{\E}{\text{E}}
\newcommand{\eps}{\varepsilon}
\newcommand{\false}{\text{false}}
\newcommand{\Gbar}{\bar{G}}
\newcommand{\Gnp}{{\cal G}(n,p)}
\newcommand{\Int}{\mathbf{I}}
\newcommand{\lambdamax}{\lambda_{\text{max}}}
\newcommand{\LPI}{$\text{LP}_{\text{I}}$}
\newcommand{\LPII}{$\text{LP}_{\text{II}}$}
\newcommand{\NN}{\mathbb{N}}
\newcommand{\om}{\omega}
\newcommand{\Om}{\Omega}
\newcommand{\OPT}{\text{OPT}}
\newcommand{\OPTf}{\text{OPT}_f}
\newcommand{\phase}{\bar{\sigma}}
\newcommand{\phat}{\hat{p}}
\newcommand{\pt}{p^{(t)}}
\newcommand{\rbot}{r^{\bot}}
\newcommand{\rhat}{\hat{r}}
\newcommand{\RENT}{\text{RENT}}
\newcommand{\RR}{\mathbb{R}}
\newcommand{\RRmn}{\mathbb{R}^{m\times n}}
\newcommand{\RRn}{\mathbb{R}^n}
\newcommand{\RRnn}{\mathbb{R}^{n\times n}}
\newcommand{\sgn}{\text{sgn}}
\newcommand{\stot}{$s$-$t$}
\newcommand{\true}{\text{true}}
\newcommand{\st}{\text{s.t.}}
\newcommand{\vivi}{v_i^Tv_i}
\newcommand{\vivj}{v_i^Tv_j}
\newcommand{\Var}{\text{Var}}
\newcommand{\xbar}{\bar{x}}
\newcommand{\xopt}{x^*}
\newcommand{\ybar}{\bar{y}}
\newcommand{\yopt}{y^*}
\newcommand{\zbar}{\bar{z}}
\newcommand{\zopt}{z^*}
\newcommand{\ZIP}{Z^*_{\text{IP}}}
\newcommand{\ZLP}{Z^*_{\text{LP}}}
\newcommand{\ZZ}{\mathbb{Z}}

\newcommand{\tD}[1]{\textrm{\em\color{aaltoGreen}{#1}}}
\newcommand{\tB}[1]{\text{\color{aaltoBlack}{#1}}}
\newcommand{\Pt}{\ensuremath{\mathrm{P}}}
\newcommand{\NP}{\ensuremath{\mathrm{NP}}}
\newcommand{\bra}{\ensuremath{\langle}}
\newcommand{\ket}{\ensuremath{\rangle}}
\newcommand{\DFT}{\ensuremath{\operatorname{DFT}}}
\newcommand{\rev}{\ensuremath{\operatorname{rev}}}
\newcommand{\quo}{\ensuremath{\operatorname{quo}}}
\newcommand{\rem}{\ensuremath{\operatorname{rem}}}
\newcommand{\take}{\ensuremath{\operatorname{\upharpoonright}}}
\newcommand{\ord}{\ensuremath{\operatorname{ord}}}
\newtheorem{fact}{Fact}
\newcommand\Edit{\mathrm{Edit}}
\newcommand\backtrack{\stackrel{\text{back-track}}{\longrightarrow}}

\newcommand\N{\mathbb{N}} % Sorry did not see \NN etc earlier and now my \N  is everywhere :(
\newcommand\R{\mathbb{R}}
\newcommand\Z{\mathbb{Z}}

%--------- definition environment commands --

\theoremstyle{definition}
\newtheorem{definition}{Definition}[section]
\newtheorem{lemma}{Lemma}[section]
\newtheorem{corollary}{Corollary}[section]
\newtheorem{theorem}{Theorem}[section]
\newtheorem{example}{Example}[section]


%--------- STUFF FOR 2021 ITERATION ---------

\usepackage{bbm}
\usepackage{float}
\usepackage{tikz}
\newcommand{\ex}{\text{E}}
\newcommand{\pr}{\text{P}}
\allowdisplaybreaks

% question title
\newcommand{\qtitle}[1]{{\color{aaltoBlue}\textbf{#1}}}

%% algorithm environment
\usepackage[ruled, vlined]{algorithm2e}
\newcommand\mycommfont[1]{\footnotesize\ttfamily\textcolor{aaltoBlue}{#1}}
\SetCommentSty{mycommfont}
\SetFuncSty{text}
\SetKwProg{Fn}{def}{:}{}
\usepackage{ulem}

\newif\ifanswer
\newif\ifgraded
\newcommand{\InsertAnswer}[1]{\vspace{1mm}{\color{aaltoBlue}\ifanswer \textbf{Solution.}#1\fi}}

%an environment to automatically hide answers. Wrap \InsertAnswer with this
\newenvironment{answer}{}

\newcommand{\createheader}[1]{
{\color{aaltoBlue}{
\LARGE\bf CS-E3190 Principles of Algorithmic Techniques \\ [2mm]
{\it #1 \ifgraded -- Graded Exercise \else -- Tutorial Exercise \fi}

}}}


\newcommand{\rulebox}[1]{
  \ifgraded

    \vspace{-3mm}
    {\color{aaltoBlue} \rule{16cm}{1pt}}

    \centering{
      \vspace{2mm}
      \fbox{\begin{minipage}{14.5cm}
          \vspace{2mm} \small \normalfont
          {Please read the following \textbf{rules} very carefully.
            \begin{itemize}[itemsep=0.5mm, leftmargin=15pt]
              \item Do not consciously search for the solution on the internet.
              \item You are allowed to discuss the problems with your classmates but you should \textbf{write the solutions yourself}.
              \item Be aware that \textbf{if plagiarism is suspected}, you could be asked to have an interview with teaching staff.
              \item The teaching staff can assist with understanding the problem statements, but will \textbf{not be giving any hints} on how to solve the exercises.
              \item The use of generative AI tools of any kind is \textbf{not allowed}.
              \item In order to ease grading, we want the solution of each problem and subproblem to start on a \textbf{new page}. If this requirement is not met, \textbf{points will be deduced}.

            \end{itemize}
          }
          \vspace{0.5mm}
        \end{minipage}}}
  \else
    \vspace{-3mm}
    {\color{aaltoBlue} \rule{16cm}{1pt}}
    \vspace{-2mm}
  \fi
}

%-----------------------------------------------
%-----------------------------------------------

\title{CS-E4500 Advanced Course in Algorithms}
\author{Aleksi Kääriäinen  \\
	Aalto University  \\
	}

\begin{document}

\date{\today}

\maketitle

\newpage

\section*{Week 7 exercices}

\begin{enumerate}
    \item
          \begin{enumerate}
              \item The outcomes for a one-qubit system are 0 and 1. The probabilities of those outcomes are:
                    \begin{align*}
                        \mathbb{P}(0) & = \left | \frac{1 - i}{2} \right |^2 = \sqrt{\frac{1 - i}{2}\cdot \frac{1 + i}{2}}^2
                        = \frac{1 + i - i - i^2}{4} = \frac{1}{2}                                                            \\
                        \mathbb{P}(1) & = \left | \frac{1 + i}{2} \right |^2 = \sqrt{\frac{1 + i}{2}\cdot \frac{1 - i}{2}}^2
                        = \frac{1 + i - i - i^2}{4} = \frac{1}{2}
                    \end{align*}
              \item The state is:
                    \begin{align*}
                        \begin{bmatrix}
                            \alpha_0 \\ \alpha_1
                        \end{bmatrix} = H|0\ket = \frac{1}{\sqrt{2}}\begin{bmatrix}
                                                                        1 & 1 \\ 1 & -1
                                                                    \end{bmatrix} \begin{bmatrix}
                                                                                      1 \\ 0
                                                                                  \end{bmatrix} = \frac{1}{\sqrt{2}} \begin{bmatrix}
                                                                                                                         1 \\ 1
                                                                                                                     \end{bmatrix}
                    \end{align*}
                    With probabilities:
                    \begin{align*}
                        \mathbb{P}(\alpha_0) = \mathbb{P}(\alpha_1) = \sqrt{\left(\frac{1}{\sqrt{2}}\right)^2}^2 = \frac{1}{2}
                    \end{align*}
              \item Working in the same basis, $|00\ket = |0\ket \otimes |0\ket = \begin{bmatrix}
                            1 & 0 & 0 & 0
                        \end{bmatrix}^T$.
                    \begin{align*}
                          & CH(H \otimes H)                                                                                                                                                      \\
                        = & CH((I \otimes H)(H \otimes I))                                                                                                                                       \\
                        = & \begin{bmatrix}
                                1 & 0 & 0                  & 0                   \\
                                0 & 1 & 0                  & 0                   \\
                                0 & 0 & \frac{1}{\sqrt{2}} & \frac{1}{\sqrt{2}}  \\
                                0 & 0 & \frac{1}{\sqrt{2}} & \frac{-1}{\sqrt{2}} \\
                            \end{bmatrix} \left( \left(\begin{bmatrix}
                                                               1 & 0 \\ 0 & 1
                                                           \end{bmatrix} \otimes \frac{1}{\sqrt{2}}\begin{bmatrix}
                                                                                                       1 & 1  \\
                                                                                                       1 & -1
                                                                                                   \end{bmatrix}\right) \left( \frac{1}{\sqrt{2}}\begin{bmatrix}
                                                                                                                                                     1 & 1  \\
                                                                                                                                                     1 & -1
                                                                                                                                                 \end{bmatrix} \otimes \begin{bmatrix}
                                                                                                                                                                           1 & 0 \\ 0 & 1
                                                                                                                                                                       \end{bmatrix} \right) \right) \\
                        = & \begin{bmatrix}
                                1 & 0 & 0                  & 0                   \\
                                0 & 1 & 0                  & 0                   \\
                                0 & 0 & \frac{1}{\sqrt{2}} & \frac{1}{\sqrt{2}}  \\
                                0 & 0 & \frac{1}{\sqrt{2}} & \frac{-1}{\sqrt{2}} \\
                            \end{bmatrix} \cdot \frac{1}{2} \begin{bmatrix}
                                                                1 & 1  & 0 & 0  \\
                                                                1 & -1 & 0 & 0  \\
                                                                0 & 0  & 1 & 1  \\
                                                                0 & 0  & 1 & -1
                                                            \end{bmatrix} \begin{bmatrix}
                                                                              1 & 0 & 1  & 0  \\
                                                                              0 & 1 & 0  & 1  \\
                                                                              1 & 0 & -1 & 0  \\
                                                                              0 & 1 & 0  & -1
                                                                          \end{bmatrix}                                                                                         \\
                        = & \frac{1}{2} \begin{bmatrix}
                                            1 & 0 & 0                  & 0                   \\
                                            0 & 1 & 0                  & 0                   \\
                                            0 & 0 & \frac{1}{\sqrt{2}} & \frac{1}{\sqrt{2}}  \\
                                            0 & 0 & \frac{1}{\sqrt{2}} & \frac{-1}{\sqrt{2}} \\
                                        \end{bmatrix} \begin{bmatrix}
                                                          1 & 1  & 1  & 1  \\
                                                          1 & -1 & 1  & -1 \\
                                                          1 & 1  & -1 & -1 \\
                                                          1 & -1 & -1 & 1  \\
                                                      \end{bmatrix}                                                                             \\
                        = & \begin{bmatrix}
                                \frac{1}{2}        & \frac{1}{2}        & \frac{1}{2}         & \frac{1}{2}         \\
                                \frac{1}{2}        & \frac{-1}{2}       & \frac{1}{2}         & \frac{-1}{2}        \\
                                \frac{1}{\sqrt{2}} & 0                  & \frac{-1}{\sqrt{2}} & 0                   \\
                                0                  & \frac{1}{\sqrt{2}} & 0                   & \frac{-1}{\sqrt{2}}
                            \end{bmatrix}
                    \end{align*}
                    Thus:
                    \begin{align*}
                        \begin{bmatrix}
                            \alpha_{00} \\ \alpha_{01} \\ \alpha_{10} \\ \alpha_{11}
                        \end{bmatrix} =
                        CH(H \otimes H)|00\ket
                        =  \begin{bmatrix}
                               \frac{1}{2}        & \frac{1}{2}        & \frac{1}{2}         & \frac{1}{2}         \\
                               \frac{1}{2}        & \frac{-1}{2}       & \frac{1}{2}         & \frac{-1}{2}        \\
                               \frac{1}{\sqrt{2}} & 0                  & \frac{-1}{\sqrt{2}} & 0                   \\
                               0                  & \frac{1}{\sqrt{2}} & 0                   & \frac{-1}{\sqrt{2}}
                           \end{bmatrix} \begin{bmatrix}
                                             1 \\ 0 \\ 0 \\ 0
                                         \end{bmatrix}
                        = \begin{bmatrix}
                              \frac{1}{2} \\ \frac{1}{2} \\ \frac{1}{\sqrt{2}} \\ 0
                          \end{bmatrix}
                    \end{align*}
                    And the two-bit outcomes have the probabilities:
                    \begin{align*}
                        \mathbb{P}(\alpha_{00}) & = \mathbb{P}(\alpha_{01}) = \left(\frac{1}{2}\right)^2 = \frac{1}{4} \\
                        \mathbb{P}(\alpha_{10}) & = \left(\frac{1}{\sqrt(2)}\right)^2 = \frac{1}{2}                    \\
                        \mathbb{P}(\alpha_{11}) & = 0
                    \end{align*}
              \item The circuit corresponding to the state is:
                    \begin{center}
                        \includegraphics[scale=0.5]{pics/circuit.png}
                    \end{center}
          \end{enumerate}
          \newpage

    \item
          \begin{enumerate}
              \item The value table of $M_3$:
                    \begin{center}
                        \begin{tabular}{c|c|c|c|c}
                            $x_0$ & $x_1$ & $x_2$ & $x_0 + x_1 + x_2$ & $M_3$ \\
                            \hline
                            1     & 1     & 1     & 3                 & 1     \\
                            1     & 1     & 0     & 2                 & 1     \\
                            1     & 0     & 1     & 2                 & 1     \\
                            1     & 0     & 0     & 1                 & 0     \\
                            0     & 1     & 1     & 2                 & 1     \\
                            0     & 1     & 0     & 1                 & 0     \\
                            0     & 0     & 1     & 1                 & 0     \\
                            0     & 0     & 0     & 0                 & 0     \\
                        \end{tabular}
                    \end{center}
              \item The function $M_3$ is logically equivalent to the formula $(x_0 \land x_1) \lor (x_0 \land x_2) \lor (x_1 \land x_2)$. The formula can be formed into a boolean circuit
                    with 8 bits. Additional bits are needed to keep track of intermediary results.
                    \begin{center}
                        \includegraphics[scale=0.25]{pics/boolean_circuit.jpg}
                    \end{center}
              \item Transforming the circuit from part (b) to a reversible boolean circuit using the gate transformations
                    from the lecture slides:
                    \begin{center}
                        \includegraphics[scale=0.25]{pics/rev_circuit.jpg}
                    \end{center}
              \item Transforming the circuit from part (c) to a quantum circuit:
                    \begin{align*}
                        G = & X^{[8|0]} X^{[8|1]} X^{[8|2]} CCX^{[8|2, 1, 0]} X^{[8|1]} X^{[8|2]} X^{[8|1]} X^{[8|3]} X^{[8|4]} \cdot \\
                            & CCX^{[8|4, 3, 1]} X^{[8|3]} X^{[8|4]} CCX^{[8|6, 5, 2]} CCX^{[8|7, 5, 3]} CCX^{[8|7, 6, 4]}
                    \end{align*}
                    Since there are 8 qubits in the register, $q = 8$, and thus:
                    \begin{align*}
                        U_{M_3} & = G^{[8|5, 6, 7, 0, 4, 3, 2, 1]} \text{, with} \\
                        W       & = ((x_0 \land x_1), (x_0 \land x_2), (x_1 \land x_2), ((x_0 \land x_1) \lor (x_0 \land x_2)))
                    \end{align*}
          \end{enumerate}
          \newpage

    \item

          \newpage

    \item

          \newpage
\end{enumerate}
\end{document}
