\documentclass[11pt,a4paper]{article}
\input{auxiliary.tex}

%-----------------------------------------------
%-----------------------------------------------

\title{CS-E4500 Advanced Course in Algorithms}
\author{Aleksi Kääriäinen  \\
	Aalto University  \\
	}

\begin{document}

\date{\today}

\maketitle

\newpage

\section*{Week 8 exercices}

\begin{enumerate}
    \item \begin{enumerate}
              \item From the week 7 lecture slides, the $CX$-gate in matrix form is:
                    \begin{align*}
                        CX = \begin{bmatrix}
                                 1 & 0 & 0 & 0 \\
                                 0 & 1 & 0 & 0 \\
                                 0 & 0 & 0 & 1 \\
                                 0 & 0 & 1 & 0
                             \end{bmatrix},
                    \end{align*}
                    and its evaluations are
                    \begin{align*}
                        CX|00\ket & = |00\ket, & CX|01\ket  & = |01\ket, \\
                        CX|10\ket & = |11\ket, & CX^|11\ket & = |10\ket
                    \end{align*}
                    The gate $CX^{[2|0, 1]}$ is the $CX$-gate with the places of the control and data bit swapped, i.e. the evaluations are
                    \begin{align*}
                        CX^{[2|0, 1]}|00\ket & = |00\ket, & CX^{[2|0, 1]}|01\ket & = |11\ket, \\
                        CX^{[2|0, 1]}|10\ket & = |10\ket, & CX^{[2|0, 1]}|11\ket & = |01\ket,
                    \end{align*} which corresponds to the matrix
                    \begin{align*}
                        CX^{[2|0, 1]} = \begin{bmatrix}
                                            1 & 0 & 0 & 0 \\
                                            0 & 0 & 0 & 1 \\
                                            0 & 0 & 1 & 0 \\
                                            0 & 1 & 0 & 0
                                        \end{bmatrix}.
                    \end{align*}
                    Calculating the two-qubit circuit matrix yields:
                    \begin{align*}
                        A = (CX)(CX)^{[2|0, 1]}(CX) = \begin{bmatrix}
                                                          1 & 0 & 0 & 0 \\
                                                          0 & 0 & 1 & 0 \\
                                                          0 & 1 & 0 & 0 \\
                                                          0 & 0 & 0 & 1 \\
                                                      \end{bmatrix}
                    \end{align*}
                    Evaluating the basis vectors in the circuit yields:
                    \begin{align*}
                        A|00\ket & = |00\ket  \\
                        A|01\ket & = |10\ket  \\
                        A|10\ket & = |01\ket  \\
                        A|11\ket & = |11\ket,
                    \end{align*}
                    i.e. the circuit flips the bits if they are different.

                    \newpage

              \item The d-bit reversal permutation matrix $P_d$ reverses the input binary word $b \in \{0, 1\}^d$ such that
                    \begin{align*}
                        P_d|b_{d - 1}b_{d-2} \cdots b_1 b_0\ket = |b_0 b_1 \cdots b_{d - 2} b_{d - 1}\ket
                    \end{align*}
                    Notice that the circuit in part (a) uses exactly $3 \frac{d}{2}$ two-qubit gates and produces
                    a reversal of the input. Applying the matrix $A$ from part (a) to each pair of bits $(j, d - 1 - j)$ for all
                    $j = 0, 1, 2, \dots \frac{d}{2} - 1$ should then produce the bit reversal of $b$. Thus, let us define
                    \begin{align*}
                        P_d = \prod_{j \in [\frac{d}{2}]} (CX)^{[d|j, d - 1 - j]} (CX)^{[d|j, d - 1 - j]} (CX)^{[d| d - 1 - j, j]}.
                    \end{align*}
                    And it has been shown that $P_d$ can be decomposed in such a way that it uses $3 \frac{d}{2} - 1$ two-qubit CX-gates.
          \end{enumerate}



          \newpage

    \item

          \newpage

    \item

          \newpage

    \item

          \newpage
\end{enumerate}
\end{document}
