\documentclass[11pt,a4paper]{article}
\input{auxiliary.tex}

%-----------------------------------------------
%-----------------------------------------------

\title{CS-E4500 Advanced Course in Algorithms}
\author{Aleksi Kääriäinen  \\
	Aalto University  \\
	}

\begin{document}

\date{\today}

\maketitle

\newpage

\section*{Week 6 exercices}

\begin{enumerate}
    \item \begin{enumerate}
              \item The matrix multiplication tensor is defined as:
                    \begin{align*}
                        (M_n)_{in + k^\prime, i^\prime n + j, j^\prime n + k} = \begin{cases}
                                                                                    1 & \text{if } i = i^\prime, j = j^\prime \text{, and }k = k^\prime \\
                                                                                    0 & \text{otherwise},
                                                                                \end{cases}
                    \end{align*} for all $i, i^\prime, j, j^\prime, k, k^\prime \in [n]$.
                    For $n = 2$, the following coordinates have the value 1:
                    \begin{center}
                        \begin{tabular}{c}
                            (0, 0, 0) \\
                            (0, 1, 2) \\
                            (1, 0, 1) \\
                            (1, 1, 3) \\
                            (2, 2, 0) \\
                            (2, 3, 2) \\
                            (3, 2, 1) \\
                            (3, 3, 3)
                        \end{tabular}
                    \end{center}
                    Thus $M_2$ is:
                    \begin{align*}
                        M_2 = \begin{bmatrix}
                                  \begin{bmatrix}
                                1 & 0 & 0 & 0 \\
                                0 & 0 & 1 & 0 \\
                                0 & 0 & 0 & 0 \\
                                0 & 0 & 0 & 0
                            \end{bmatrix} \\
                                  \begin{bmatrix}
                                0 & 1 & 0 & 0 \\
                                0 & 0 & 0 & 1 \\
                                0 & 0 & 0 & 0 \\
                                0 & 0 & 0 & 0
                            \end{bmatrix} \\
                                  \begin{bmatrix}
                                0 & 0 & 0 & 0 \\
                                0 & 0 & 0 & 0 \\
                                1 & 0 & 0 & 0 \\
                                0 & 0 & 1 & 0
                            \end{bmatrix} \\
                                  \begin{bmatrix}
                                0 & 0 & 0 & 0 \\
                                0 & 0 & 0 & 0 \\
                                0 & 1 & 0 & 0 \\
                                0 & 0 & 0 & 1
                            \end{bmatrix}
                              \end{bmatrix}
                    \end{align*}
                    \newpage
              \item Calculating the Kruskal product $K(A, B, C)$ with the following code:

                    \begin{lstlisting}[language=Python]
import numpy as np
                        
def kruskal(A, B, C, n, r):
    T = np.zeros((n, n, n))
    for i in range(n):
        for j in range(n):
            for k in range(n):
                for l in range(r):
                    T[i][j][k] += A[i][l] \
                        * B[j][l] * C[k][l]
    return T.astype(int)
                    \end{lstlisting}
                    yields:

                    \begin{lstlisting}[language=Python]
array([[[1, 0, 0, 0],
        [0, 0, 1, 0],
        [0, 0, 0, 0],
        [0, 0, 0, 0]],
            
       [[0, 1, 0, 0],
        [0, 0, 0, 1],
        [0, 0, 0, 0],
        [0, 0, 0, 0]],
            
       [[0, 0, 0, 0],
        [0, 0, 0, 0],
        [1, 0, 0, 0],
        [0, 0, 1, 0]],
            
       [[0, 0, 0, 0],
        [0, 0, 0, 0],
        [0, 1, 0, 0],
        [0, 0, 0, 1]]])
                \end{lstlisting}
                    Which is equivalent to the result in part (a). Thus $K(A, B, C) = M_2$


              \item Matrix product:
                    \begin{align*}
                        W & = UV                                            \\
                          & = \begin{bmatrix}
                                  1 & 2 \\ 3 & 4
                              \end{bmatrix} \begin{bmatrix}
                                                5 & 6 \\ 7 & 8
                                            \end{bmatrix}                  \\
                          & = \begin{bmatrix}
                                  5 \cdot 1 + 7 \cdot 2 & 6 \cdot 1 + 8 \cdot 2 \\
                                  5 \cdot 3 + 7 \cdot 4 & 6 \cdot 3 + 8 \cdot 4
                              \end{bmatrix} \\
                          & = \begin{bmatrix}
                                  19 & 22 \\ 43 & 50
                              \end{bmatrix}
                    \end{align*}

              \item Vectorizations:
                    \begin{align*}
                        \underline{U} = \begin{bmatrix}
                                            1 \\ 2 \\ 3 \\ 4
                                        \end{bmatrix},
                        \underline{V} = \begin{bmatrix}
                                            5 \\ 6 \\ 7 \\ 8
                                        \end{bmatrix},
                        \underline{W} = \begin{bmatrix}
                                            19 \\ 22 \\ 43 \\ 50
                                        \end{bmatrix}
                    \end{align*}
              \item Bilinear map:
                    \begin{align*}
                         & \underline{U} = u, \underline{V} = v                          \\
                         & w_i = \sum_{j \in [4]}\sum_{k \in [4]}M_{2_{i, j, k}} u_j v_k \\
                         & w = \begin{bmatrix}
                                   1 \cdot 1 \cdot 5 + 1 \cdot 2 \cdot 7 \\
                                   1 \cdot 1 \cdot 6 + 1 \cdot 2 \cdot 8 \\
                                   1 \cdot 3 \cdot 5 + 1 \cdot 4 \cdot 7 \\
                                   1 \cdot 3 \cdot 6 + 1 \cdot 4 \cdot 8
                               \end{bmatrix}                     \\
                         & w = \begin{bmatrix}
                                   19 \\ 22 \\ 43 \\ 50
                               \end{bmatrix} = \underline{W}
                    \end{align*}
                    Hadamard:
                    \begin{align*}
                          & A((B^T u)\odot (C^T v))                                         \\
                        = & A \left ( \begin{bmatrix}
                                          7 \\ 2 \\ 6 \\ 9 \\ 10 \\ 3 \\ 1
                                      \end{bmatrix} \odot \begin{bmatrix}
                                                              15 \\ 7 \\ -2 \\ 9 \\ 6 \\ -4 \\ 5
                                                          \end{bmatrix} \right ) \\
                        = & A \begin{bmatrix}
                                  105 & 14 & -12 & 81 & 60 & -12 & 5
                              \end{bmatrix}^T \\
                        = & \begin{bmatrix}
                            19 \\ 22 \\43 \\ 50
                        \end{bmatrix} = \underline{W}
                    \end{align*}
          \end{enumerate}

          \newpage

    \item

          \newpage

    \item

          \newpage

    \item

          \newpage
\end{enumerate}
\end{document}
