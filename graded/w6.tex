\documentclass[11pt,a4paper]{article}
\usepackage{hyperref}
\usepackage[left=1.5in,right=1.5in,top=1.5in,bottom=1.5in]{geometry}
\usepackage{enumitem}
\usepackage{amssymb,amsmath,mathrsfs,amsthm}
\usepackage{parskip, graphicx, float, wrapfig, subcaption}


\setlength{\parindent}{0pt}
\setlength{\parskip}{3mm}

\pagestyle{empty}

%Title and formatting stuff
\usepackage{titlesec}
\titleformat{\chapter}
{\normalfont\Huge\color{aaltoBlue}}{\thechapter}{20pt}{\filleft\Huge}[\vskip 0.5ex\titlerule]
\usepackage{fancyhdr}
\pagestyle{fancy}
\fancyhead[L]{}
\newcommand{\mymarginpar}[1]{\marginpar{\it \small \raggedleft #1}}
\renewcommand*\thesection{\arabic{section}}

% Miscellaneous LaTeX macros
\newcommand{\heading}[1]{{\Large\color{blue!80!black}\textbf{#1}}\par}
\newcommand{\slemp}[1]{{\color{red}#1}}
\newcommand{\slred}[1]{{\color{red}#1}}
\newcommand{\slbf}[1]{{\color{blue}\textbf{#1}}}
\newcommand{\pair}[2]{\langle {#1}, {#2} \rangle}
\newcommand{\Abar}{\bar{A}}
\newcommand{\ALG}{\text{ALG}}
\newcommand{\approxA}{\alpha_{\cal A}}
\newcommand{\argmin}{\operatornamewithlimits{arg\ min}}
\newcommand{\assign}{\mbox{$\; \leftarrow \;$}}
\newcommand{\bbar}{\bar{b}}
\newcommand{\Bin}{\text{Bin}}
\newcommand{\calA}{{\cal A}}
\newcommand{\calC}{{\cal C}}
\newcommand{\calI}{{\cal I}}
\newcommand{\calM}{{\cal M}}
\newcommand{\calN}{{\cal N}}
\newcommand{\calS}{{\cal S}}
\newcommand{\chat}{\hat{c}}
\newcommand{\chatopt}{\hat{c}^*}
\newcommand{\copt}{c^*}
\newcommand{\Cov}{\text{Cov}}
\newcommand{\DC}{\text{DC}}
\newcommand{\delC}{\partial C}
\newcommand{\delS}{\partial S}
\newcommand{\delT}{\partial T}
\newcommand{\diag}{\textit{diag}}
\newcommand{\dtv}{d_{\text{V}}}
\newcommand{\dom}{\mathcal{D}}
\newcommand{\E}{\text{E}}
\newcommand{\eps}{\varepsilon}
\newcommand{\false}{\text{false}}
\newcommand{\Gbar}{\bar{G}}
\newcommand{\Gnp}{{\cal G}(n,p)}
\newcommand{\Int}{\mathbf{I}}
\newcommand{\lambdamax}{\lambda_{\text{max}}}
\newcommand{\LPI}{$\text{LP}_{\text{I}}$}
\newcommand{\LPII}{$\text{LP}_{\text{II}}$}
\newcommand{\NN}{\mathbb{N}}
\newcommand{\om}{\omega}
\newcommand{\Om}{\Omega}
\newcommand{\OPT}{\text{OPT}}
\newcommand{\OPTf}{\text{OPT}_f}
\newcommand{\phase}{\bar{\sigma}}
\newcommand{\phat}{\hat{p}}
\newcommand{\pt}{p^{(t)}}
\newcommand{\rbot}{r^{\bot}}
\newcommand{\rhat}{\hat{r}}
\newcommand{\RENT}{\text{RENT}}
\newcommand{\RR}{\mathbb{R}}
\newcommand{\RRmn}{\mathbb{R}^{m\times n}}
\newcommand{\RRn}{\mathbb{R}^n}
\newcommand{\RRnn}{\mathbb{R}^{n\times n}}
\newcommand{\sgn}{\text{sgn}}
\newcommand{\stot}{$s$-$t$}
\newcommand{\true}{\text{true}}
\newcommand{\st}{\text{s.t.}}
\newcommand{\vivi}{v_i^Tv_i}
\newcommand{\vivj}{v_i^Tv_j}
\newcommand{\Var}{\text{Var}}
\newcommand{\xbar}{\bar{x}}
\newcommand{\xopt}{x^*}
\newcommand{\ybar}{\bar{y}}
\newcommand{\yopt}{y^*}
\newcommand{\zbar}{\bar{z}}
\newcommand{\zopt}{z^*}
\newcommand{\ZIP}{Z^*_{\text{IP}}}
\newcommand{\ZLP}{Z^*_{\text{LP}}}
\newcommand{\ZZ}{\mathbb{Z}}

\newcommand{\tD}[1]{\textrm{\em\color{aaltoGreen}{#1}}}
\newcommand{\tB}[1]{\text{\color{aaltoBlack}{#1}}}
\newcommand{\Pt}{\ensuremath{\mathrm{P}}}
\newcommand{\NP}{\ensuremath{\mathrm{NP}}}
\newcommand{\bra}{\ensuremath{\langle}}
\newcommand{\ket}{\ensuremath{\rangle}}
\newcommand{\DFT}{\ensuremath{\operatorname{DFT}}}
\newcommand{\rev}{\ensuremath{\operatorname{rev}}}
\newcommand{\quo}{\ensuremath{\operatorname{quo}}}
\newcommand{\rem}{\ensuremath{\operatorname{rem}}}
\newcommand{\take}{\ensuremath{\operatorname{\upharpoonright}}}
\newcommand{\ord}{\ensuremath{\operatorname{ord}}}
\newtheorem{fact}{Fact}
\newcommand\Edit{\mathrm{Edit}}
\newcommand\backtrack{\stackrel{\text{back-track}}{\longrightarrow}}

\newcommand\N{\mathbb{N}} % Sorry did not see \NN etc earlier and now my \N  is everywhere :(
\newcommand\R{\mathbb{R}}
\newcommand\Z{\mathbb{Z}}

%--------- definition environment commands --

\theoremstyle{definition}
\newtheorem{definition}{Definition}[section]
\newtheorem{lemma}{Lemma}[section]
\newtheorem{corollary}{Corollary}[section]
\newtheorem{theorem}{Theorem}[section]
\newtheorem{example}{Example}[section]


%--------- STUFF FOR 2021 ITERATION ---------

\usepackage{bbm}
\usepackage{float}
\usepackage{tikz}
\newcommand{\ex}{\text{E}}
\newcommand{\pr}{\text{P}}
\allowdisplaybreaks

% question title
\newcommand{\qtitle}[1]{{\color{aaltoBlue}\textbf{#1}}}

%% algorithm environment
\usepackage[ruled, vlined]{algorithm2e}
\newcommand\mycommfont[1]{\footnotesize\ttfamily\textcolor{aaltoBlue}{#1}}
\SetCommentSty{mycommfont}
\SetFuncSty{text}
\SetKwProg{Fn}{def}{:}{}
\usepackage{ulem}

\newif\ifanswer
\newif\ifgraded
\newcommand{\InsertAnswer}[1]{\vspace{1mm}{\color{aaltoBlue}\ifanswer \textbf{Solution.}#1\fi}}

%an environment to automatically hide answers. Wrap \InsertAnswer with this
\newenvironment{answer}{}

\newcommand{\createheader}[1]{
{\color{aaltoBlue}{
\LARGE\bf CS-E3190 Principles of Algorithmic Techniques \\ [2mm]
{\it #1 \ifgraded -- Graded Exercise \else -- Tutorial Exercise \fi}

}}}


\newcommand{\rulebox}[1]{
  \ifgraded

    \vspace{-3mm}
    {\color{aaltoBlue} \rule{16cm}{1pt}}

    \centering{
      \vspace{2mm}
      \fbox{\begin{minipage}{14.5cm}
          \vspace{2mm} \small \normalfont
          {Please read the following \textbf{rules} very carefully.
            \begin{itemize}[itemsep=0.5mm, leftmargin=15pt]
              \item Do not consciously search for the solution on the internet.
              \item You are allowed to discuss the problems with your classmates but you should \textbf{write the solutions yourself}.
              \item Be aware that \textbf{if plagiarism is suspected}, you could be asked to have an interview with teaching staff.
              \item The teaching staff can assist with understanding the problem statements, but will \textbf{not be giving any hints} on how to solve the exercises.
              \item The use of generative AI tools of any kind is \textbf{not allowed}.
              \item In order to ease grading, we want the solution of each problem and subproblem to start on a \textbf{new page}. If this requirement is not met, \textbf{points will be deduced}.

            \end{itemize}
          }
          \vspace{0.5mm}
        \end{minipage}}}
  \else
    \vspace{-3mm}
    {\color{aaltoBlue} \rule{16cm}{1pt}}
    \vspace{-2mm}
  \fi
}

%-----------------------------------------------
%-----------------------------------------------

\title{CS-E4500 Advanced Course in Algorithms}
\author{Aleksi Kääriäinen  \\
	Aalto University  \\
	}

\begin{document}

\date{\today}

\maketitle

\newpage

\section*{Week 6 exercices}

\begin{enumerate}
    \item \begin{enumerate}
              \item The matrix multiplication tensor is defined as:
                    \begin{align*}
                        (M_n)_{in + k^\prime, i^\prime n + j, j^\prime n + k} = \begin{cases}
                                                                                    1 & \text{if } i = i^\prime, j = j^\prime \text{, and }k = k^\prime \\
                                                                                    0 & \text{otherwise},
                                                                                \end{cases}
                    \end{align*} for all $i, i^\prime, j, j^\prime, k, k^\prime \in [n]$.
                    For $n = 2$, the following coordinates have the value 1:
                    \begin{center}
                        \begin{tabular}{c}
                            (0, 0, 0) \\
                            (0, 1, 2) \\
                            (1, 0, 1) \\
                            (1, 1, 3) \\
                            (2, 2, 0) \\
                            (2, 3, 2) \\
                            (3, 2, 1) \\
                            (3, 3, 3)
                        \end{tabular}
                    \end{center}
                    Thus $M_2$ is:
                    \begin{align*}
                        M_2 = \begin{bmatrix}
                                  \begin{bmatrix}
                                1 & 0 & 0 & 0 \\
                                0 & 0 & 1 & 0 \\
                                0 & 0 & 0 & 0 \\
                                0 & 0 & 0 & 0
                            \end{bmatrix} \\
                                  \begin{bmatrix}
                                0 & 1 & 0 & 0 \\
                                0 & 0 & 0 & 1 \\
                                0 & 0 & 0 & 0 \\
                                0 & 0 & 0 & 0
                            \end{bmatrix} \\
                                  \begin{bmatrix}
                                0 & 0 & 0 & 0 \\
                                0 & 0 & 0 & 0 \\
                                1 & 0 & 0 & 0 \\
                                0 & 0 & 1 & 0
                            \end{bmatrix} \\
                                  \begin{bmatrix}
                                0 & 0 & 0 & 0 \\
                                0 & 0 & 0 & 0 \\
                                0 & 1 & 0 & 0 \\
                                0 & 0 & 0 & 1
                            \end{bmatrix}
                              \end{bmatrix}
                    \end{align*}
                    \newpage
              \item Calculating the Kruskal product $K(A, B, C)$ with the following code:

                    \begin{lstlisting}[language=Python]
import numpy as np
                        
def kruskal(A, B, C, n, r):
    T = np.zeros((n, n, n))
    for i in range(n):
        for j in range(n):
            for k in range(n):
                for l in range(r):
                    T[i][j][k] += A[i][l] \
                        * B[j][l] * C[k][l]
    return T.astype(int)

kruskal(A, B, C, 4, 7)

                    \end{lstlisting}

                    yields:

                    \begin{lstlisting}[language=Python]
array([[[1, 0, 0, 0],
        [0, 0, 1, 0],
        [0, 0, 0, 0],
        [0, 0, 0, 0]],
            
       [[0, 1, 0, 0],
        [0, 0, 0, 1],
        [0, 0, 0, 0],
        [0, 0, 0, 0]],
            
       [[0, 0, 0, 0],
        [0, 0, 0, 0],
        [1, 0, 0, 0],
        [0, 0, 1, 0]],
            
       [[0, 0, 0, 0],
        [0, 0, 0, 0],
        [0, 1, 0, 0],
        [0, 0, 0, 1]]])
                \end{lstlisting}
                    Which is equivalent to the result in part (a). Thus $K(A, B, C) = M_2$


              \item Matrix product:
                    \begin{align*}
                        W & = UV                                            \\
                          & = \begin{bmatrix}
                                  1 & 2 \\ 3 & 4
                              \end{bmatrix} \begin{bmatrix}
                                                5 & 6 \\ 7 & 8
                                            \end{bmatrix}                  \\
                          & = \begin{bmatrix}
                                  5 \cdot 1 + 7 \cdot 2 & 6 \cdot 1 + 8 \cdot 2 \\
                                  5 \cdot 3 + 7 \cdot 4 & 6 \cdot 3 + 8 \cdot 4
                              \end{bmatrix} \\
                          & = \begin{bmatrix}
                                  19 & 22 \\ 43 & 50
                              \end{bmatrix}
                    \end{align*}

              \item Vectorizations:
                    \begin{align*}
                        \underline{U} = \begin{bmatrix}
                                            1 \\ 2 \\ 3 \\ 4
                                        \end{bmatrix},
                        \underline{V} = \begin{bmatrix}
                                            5 \\ 6 \\ 7 \\ 8
                                        \end{bmatrix},
                        \underline{W} = \begin{bmatrix}
                                            19 \\ 22 \\ 43 \\ 50
                                        \end{bmatrix}
                    \end{align*}
              \item Bilinear map:
                    \begin{align*}
                         & \underline{U} = u, \underline{V} = v                          \\
                         & w_i = \sum_{j \in [4]}\sum_{k \in [4]}M_{2_{i, j, k}} u_j v_k \\
                         & w = \begin{bmatrix}
                                   1 \cdot 1 \cdot 5 + 1 \cdot 2 \cdot 7 \\
                                   1 \cdot 1 \cdot 6 + 1 \cdot 2 \cdot 8 \\
                                   1 \cdot 3 \cdot 5 + 1 \cdot 4 \cdot 7 \\
                                   1 \cdot 3 \cdot 6 + 1 \cdot 4 \cdot 8
                               \end{bmatrix}                     \\
                         & w = \begin{bmatrix}
                                   19 \\ 22 \\ 43 \\ 50
                               \end{bmatrix} = \underline{W}
                    \end{align*}
                    Hadamard:
                    \begin{align*}
                          & A((B^T u)\odot (C^T v))                                         \\
                        = & A \left ( \begin{bmatrix}
                                              7 \\ 2 \\ 6 \\ 9 \\ 10 \\ 3 \\ 1
                                          \end{bmatrix} \odot \begin{bmatrix}
                                                                  15 \\ 7 \\ -2 \\ 9 \\ 6 \\ -4 \\ 5
                                                              \end{bmatrix} \right ) \\
                        = & A \begin{bmatrix}
                                  105 & 14 & -12 & 81 & 60 & -12 & 5
                              \end{bmatrix}^T                            \\
                        = & \begin{bmatrix}
                                19 \\ 22 \\43 \\ 50
                            \end{bmatrix} = \underline{W}
                    \end{align*}
          \end{enumerate}
          \newpage

    \item \begin{enumerate}
              \item Let $A = [a_{i, h}], B = [b_{h, j}]$. By definition of Kronecker product, $A \otimes C = [a_{i, h}C]$ and
                    $B \otimes D = [b_{h, j}D]$. The $i, j$th block of $(A \otimes C)(B \otimes D)$ is:
                    \begin{align*}
                        ((A \otimes C)(B \otimes D))_{i, j} & = \sum_{h \in [n]}(a_{i, h}C)(b_{h, j}D) \\
                                                            & = \sum_{h \in [n]}(a_{i, h}b_{h, j})(CD) \\
                                                            & = (AB)_{i, j} CD                         \\
                                                            & = ((AB) \otimes (CD))_{i, j}
                    \end{align*}
                    The result is immediate from the definition of the Kronecker Product.

              \item $K(A, B, C) \in \mathbb{F}^{m \times n \times p}$. Let $i \in [m], j \in [n], k \in [p]$. Similarly,
                    $K(A^\prime, B^\prime, C^\prime) \in \mathbb{F}^{m^\prime \times n^\prime \times p^\prime}$.
                    Let $i^\prime \in [m^\prime], j^\prime \in [n^\prime], k^\prime \in [p^\prime]$.
                    Then, the Kronecker product
                    \begin{align*}
                          & (K(A, B, C) \otimes K(A^\prime, B^\prime, C^\prime))_{i m^\prime + i^\prime, j n^\prime + j^\prime, k p^\prime + k^\prime}          \\
                        = & K(A, B, C)_{i, j, k} \cdot K(A^\prime, B^\prime, C^\prime)_{i^\prime, j^\prime, k^\prime}                                           \\
                        = & \sum_{l \in [r]} A_{i, l} B_{j, l} C_{k, l} \cdot
                        \sum_{l^\prime \in [r^\prime]}  A_{i^\prime, l^\prime}^\prime B_{j^\prime, l^\prime}^\prime C_{k^\prime, l^\prime}^\prime               \\
                        = & \sum_{l \in [r]} \sum_{l^\prime \in [r^\prime]} A_{i, l}A^\prime_{i^\prime, l^\prime} \cdot B_{j, l}B^\prime_{j^\prime, l^\prime}
                        \cdot C_{k,l}C^\prime_{k^\prime, l^\prime}                                                                                              \\
                        = & \sum_{l \in [r]} \sum_{l^\prime \in [r^\prime]} (A \otimes A^\prime)_{i m^\prime + i^\prime, l r^\prime + l^\prime} \cdot
                        (B \otimes B^\prime)_{j m^\prime + j^\prime, l r^\prime + l^\prime} \cdot
                        (C \otimes C^\prime)_{k p^\prime + k^\prime, l r^\prime + l^\prime}                                                                     \\
                        = & K(A \otimes A^\prime, B \otimes B^\prime, C \otimes C^\prime)_{i m^\prime + i^\prime, j n^\prime + j^\prime, k p^\prime + k^\prime}
                    \end{align*}
              \item For all $i \in [m]$, $w = T[u, v]$ is defined as:
                    \begin{align*}
                        w_i & = \sum_{j \in [n]} \sum_{k \in [p]} T_{i, j, k} u_j v_k                                           \\
                            & = \sum_{j \in [n]} \sum_{k \in [p]} K(A, B, C)_{i, j, k}  u_j v_k                                 \\
                            & = \sum_{j \in [n]} \sum_{k \in [p]} \sum_{l \in [r]} A_{i, l} B_{j, l} C_{k, l} u_j v_k           \\
                            & = \sum_{j \in [n]} \sum_{k \in [p]} \sum_{l \in [r]} A_{i, l} ((B_{l, j}^T u_j) \cdot (C_{l, k}^T v_k)) \\
                        w   & = A((B^T u) \odot (C^T v))
                    \end{align*}

          \end{enumerate}

          \newpage

    \item

          \newpage

    \item

          \newpage
\end{enumerate}
\end{document}
