\documentclass[11pt,a4paper]{article}
\usepackage{hyperref}
\usepackage[left=1.5in,right=1.5in,top=1.5in,bottom=1.5in]{geometry}
\usepackage{enumitem}
\usepackage{amssymb,amsmath,mathrsfs,amsthm}
\usepackage{parskip, graphicx, float, wrapfig, subcaption}


\setlength{\parindent}{0pt}
\setlength{\parskip}{3mm}

\pagestyle{empty}

%Title and formatting stuff
\usepackage{titlesec}
\titleformat{\chapter}
{\normalfont\Huge\color{aaltoBlue}}{\thechapter}{20pt}{\filleft\Huge}[\vskip 0.5ex\titlerule]
\usepackage{fancyhdr}
\pagestyle{fancy}
\fancyhead[L]{}
\newcommand{\mymarginpar}[1]{\marginpar{\it \small \raggedleft #1}}
\renewcommand*\thesection{\arabic{section}}

% Miscellaneous LaTeX macros
\newcommand{\heading}[1]{{\Large\color{blue!80!black}\textbf{#1}}\par}
\newcommand{\slemp}[1]{{\color{red}#1}}
\newcommand{\slred}[1]{{\color{red}#1}}
\newcommand{\slbf}[1]{{\color{blue}\textbf{#1}}}
\newcommand{\pair}[2]{\langle {#1}, {#2} \rangle}
\newcommand{\Abar}{\bar{A}}
\newcommand{\ALG}{\text{ALG}}
\newcommand{\approxA}{\alpha_{\cal A}}
\newcommand{\argmin}{\operatornamewithlimits{arg\ min}}
\newcommand{\assign}{\mbox{$\; \leftarrow \;$}}
\newcommand{\bbar}{\bar{b}}
\newcommand{\Bin}{\text{Bin}}
\newcommand{\calA}{{\cal A}}
\newcommand{\calC}{{\cal C}}
\newcommand{\calI}{{\cal I}}
\newcommand{\calM}{{\cal M}}
\newcommand{\calN}{{\cal N}}
\newcommand{\calS}{{\cal S}}
\newcommand{\chat}{\hat{c}}
\newcommand{\chatopt}{\hat{c}^*}
\newcommand{\copt}{c^*}
\newcommand{\Cov}{\text{Cov}}
\newcommand{\DC}{\text{DC}}
\newcommand{\delC}{\partial C}
\newcommand{\delS}{\partial S}
\newcommand{\delT}{\partial T}
\newcommand{\diag}{\textit{diag}}
\newcommand{\dtv}{d_{\text{V}}}
\newcommand{\dom}{\mathcal{D}}
\newcommand{\E}{\text{E}}
\newcommand{\eps}{\varepsilon}
\newcommand{\false}{\text{false}}
\newcommand{\Gbar}{\bar{G}}
\newcommand{\Gnp}{{\cal G}(n,p)}
\newcommand{\Int}{\mathbf{I}}
\newcommand{\lambdamax}{\lambda_{\text{max}}}
\newcommand{\LPI}{$\text{LP}_{\text{I}}$}
\newcommand{\LPII}{$\text{LP}_{\text{II}}$}
\newcommand{\NN}{\mathbb{N}}
\newcommand{\om}{\omega}
\newcommand{\Om}{\Omega}
\newcommand{\OPT}{\text{OPT}}
\newcommand{\OPTf}{\text{OPT}_f}
\newcommand{\phase}{\bar{\sigma}}
\newcommand{\phat}{\hat{p}}
\newcommand{\pt}{p^{(t)}}
\newcommand{\rbot}{r^{\bot}}
\newcommand{\rhat}{\hat{r}}
\newcommand{\RENT}{\text{RENT}}
\newcommand{\RR}{\mathbb{R}}
\newcommand{\RRmn}{\mathbb{R}^{m\times n}}
\newcommand{\RRn}{\mathbb{R}^n}
\newcommand{\RRnn}{\mathbb{R}^{n\times n}}
\newcommand{\sgn}{\text{sgn}}
\newcommand{\stot}{$s$-$t$}
\newcommand{\true}{\text{true}}
\newcommand{\st}{\text{s.t.}}
\newcommand{\vivi}{v_i^Tv_i}
\newcommand{\vivj}{v_i^Tv_j}
\newcommand{\Var}{\text{Var}}
\newcommand{\xbar}{\bar{x}}
\newcommand{\xopt}{x^*}
\newcommand{\ybar}{\bar{y}}
\newcommand{\yopt}{y^*}
\newcommand{\zbar}{\bar{z}}
\newcommand{\zopt}{z^*}
\newcommand{\ZIP}{Z^*_{\text{IP}}}
\newcommand{\ZLP}{Z^*_{\text{LP}}}
\newcommand{\ZZ}{\mathbb{Z}}

\newcommand{\tD}[1]{\textrm{\em\color{aaltoGreen}{#1}}}
\newcommand{\tB}[1]{\text{\color{aaltoBlack}{#1}}}
\newcommand{\Pt}{\ensuremath{\mathrm{P}}}
\newcommand{\NP}{\ensuremath{\mathrm{NP}}}
\newcommand{\bra}{\ensuremath{\langle}}
\newcommand{\ket}{\ensuremath{\rangle}}
\newcommand{\DFT}{\ensuremath{\operatorname{DFT}}}
\newcommand{\rev}{\ensuremath{\operatorname{rev}}}
\newcommand{\quo}{\ensuremath{\operatorname{quo}}}
\newcommand{\rem}{\ensuremath{\operatorname{rem}}}
\newcommand{\take}{\ensuremath{\operatorname{\upharpoonright}}}
\newcommand{\ord}{\ensuremath{\operatorname{ord}}}
\newtheorem{fact}{Fact}
\newcommand\Edit{\mathrm{Edit}}
\newcommand\backtrack{\stackrel{\text{back-track}}{\longrightarrow}}

\newcommand\N{\mathbb{N}} % Sorry did not see \NN etc earlier and now my \N  is everywhere :(
\newcommand\R{\mathbb{R}}
\newcommand\Z{\mathbb{Z}}

%--------- definition environment commands --

\theoremstyle{definition}
\newtheorem{definition}{Definition}[section]
\newtheorem{lemma}{Lemma}[section]
\newtheorem{corollary}{Corollary}[section]
\newtheorem{theorem}{Theorem}[section]
\newtheorem{example}{Example}[section]


%--------- STUFF FOR 2021 ITERATION ---------

\usepackage{bbm}
\usepackage{float}
\usepackage{tikz}
\newcommand{\ex}{\text{E}}
\newcommand{\pr}{\text{P}}
\allowdisplaybreaks

% question title
\newcommand{\qtitle}[1]{{\color{aaltoBlue}\textbf{#1}}}

%% algorithm environment
\usepackage[ruled, vlined]{algorithm2e}
\newcommand\mycommfont[1]{\footnotesize\ttfamily\textcolor{aaltoBlue}{#1}}
\SetCommentSty{mycommfont}
\SetFuncSty{text}
\SetKwProg{Fn}{def}{:}{}
\usepackage{ulem}

\newif\ifanswer
\newif\ifgraded
\newcommand{\InsertAnswer}[1]{\vspace{1mm}{\color{aaltoBlue}\ifanswer \textbf{Solution.}#1\fi}}

%an environment to automatically hide answers. Wrap \InsertAnswer with this
\newenvironment{answer}{}

\newcommand{\createheader}[1]{
{\color{aaltoBlue}{
\LARGE\bf CS-E3190 Principles of Algorithmic Techniques \\ [2mm]
{\it #1 \ifgraded -- Graded Exercise \else -- Tutorial Exercise \fi}

}}}


\newcommand{\rulebox}[1]{
  \ifgraded

    \vspace{-3mm}
    {\color{aaltoBlue} \rule{16cm}{1pt}}

    \centering{
      \vspace{2mm}
      \fbox{\begin{minipage}{14.5cm}
          \vspace{2mm} \small \normalfont
          {Please read the following \textbf{rules} very carefully.
            \begin{itemize}[itemsep=0.5mm, leftmargin=15pt]
              \item Do not consciously search for the solution on the internet.
              \item You are allowed to discuss the problems with your classmates but you should \textbf{write the solutions yourself}.
              \item Be aware that \textbf{if plagiarism is suspected}, you could be asked to have an interview with teaching staff.
              \item The teaching staff can assist with understanding the problem statements, but will \textbf{not be giving any hints} on how to solve the exercises.
              \item The use of generative AI tools of any kind is \textbf{not allowed}.
              \item In order to ease grading, we want the solution of each problem and subproblem to start on a \textbf{new page}. If this requirement is not met, \textbf{points will be deduced}.

            \end{itemize}
          }
          \vspace{0.5mm}
        \end{minipage}}}
  \else
    \vspace{-3mm}
    {\color{aaltoBlue} \rule{16cm}{1pt}}
    \vspace{-2mm}
  \fi
}

%-----------------------------------------------
%-----------------------------------------------

\title{CS-E4500 Advanced Course in Algorithms}
\author{Aleksi Kääriäinen  \\
	Aalto University  \\
	}

\begin{document}

\date{\today}

\maketitle

\newpage

\section*{Week 5 exercices}

\begin{enumerate}
    \item Recall that a polynomial $f \in F[x]$ can be written as a product of linear factors $f = \prod_{i}(x - \xi_i)$,
          where each $\xi_i$ is a root of $f$. Assume that a polynomial of degree at most $d$ has $d + 1$ distinct roots. Then,
          \begin{align*}
              f = \prod_{i = 0}^{d + 1}(x - \xi_i)
          \end{align*}
          This clearly shows a contradiction, since $f$ is of degree $d + 1$. Thus the number of distinct roots for a polynomial
          of degree $d$ is at most $d$.
          \newpage

    \item \begin{enumerate}
              \item Let the datavector $\Phi = (7, 6, 5, 4, 3) \in \mathbb{F}^5_{11}$, and the evaluation points
                    $\Xi = (0, 1, 2, 3, 4, 5, 6) \in \mathbb{F}_{11}^7$. Construct a polynomial $f$ from the points of $\Phi$:
                    \begin{align*}
                        f = 7 + 6x + 5x^2 + 4x^3 + 3x^4
                    \end{align*}
                    The encoded representation of $\Phi$ is $f(\Xi)$:
                    \begin{align*}
                        \Psi & = (f(\xi_0), f(\xi_1), \dots f(\xi_i)) \\
                        \Psi & = (7, 3, 9, 3, 2, 7, 3)
                    \end{align*}

              \item Let $\Xi = (1, 2, 3, 4, 5, 6) \in \mathbb{F}_{13}^6$ and $\Gamma = (3, 8, 6, 0, 7, 1) \in \mathbb{F}_{13}^6$.
                    Using Gao's decoding algorithm:
                    \begin{align*}
                        g_0 & = \prod_{i = 1}^{e}(x - \xi_i)              \\
                            & = 5 + 4x + 12x^2 + 6x^3 + 6x^4 + 5x^5 + x^6
                    \end{align*}
                    Interpolating $g_1$ with $V_\Xi^{-1} \cdot \Gamma$:
                    \[ g_1 = 7 + x + 9x^3 + 5x^4 + 7x^5\]
                    Executing the Extended Euclidean algorithm yields:
                    \begin{center}
                        \resizebox{\linewidth}{!}{
                            \begin{tabular}{l|l|l|l|l}
                                $i$ & $g_i$                                       & $s_i$                        & $t_i$                                 & $q_i$     \\
                                \hline
                                0   & $x^6 + 5x^5 + 6x^4 + 6x^3 + 12x^2 + 4x + 5$ & 1                            & 0                                     & 0         \\
                                1   & $7x^5 + 5x^4 + 9x^3 + x + 7$                & 0                            & 1                                     & $2x + 3$  \\
                                2   & $12x^4 + 5x^3 + 10x^2 + 10$                 & 1                            & $11x + 10$                            & $6x + 12$ \\
                                3   & $6x^3 + 10x^2 + 6x + 4     $                & $7x + 1$                     & $12x^2 + 3x + 11$                     & $2x + 4$  \\
                                4   & $10x^2 + 7x + 7$                            & $12x^2 + 9x + 10$            & $2x^3 + 11x^2 + 3x + 5$               & $11x + 5$ \\
                                5   & $11x + 8$                                   & $11x^3 + 10x^2 + 8x + 3$     & $4x^4 + 12x^3 + 2x^2 + 11x + 12$      & $8x + 9$  \\
                                6   & 0                                           & $3x^4 + 3x^3 + x^2 + 4x + 9$ & $7x^5 + 11x^4 + 8x^3 + 9x^2 + 3x + 1$ & 0         \\
                            \end{tabular}
                        }
                    \end{center}
                    Since $e = 6$ and $d = 1$, $D =(e + d + 1)/2 = 4$. From this, we obtain $h = 2$. Divide $g_{h + 1}$ by $t_{h + 1}$:
                    \begin{align*}
                        f_i & = g_{h + 1} \quo{t_{h + 1}} \\
                            & = 11 + 7x
                    \end{align*}
                    Since $g_{h + 1} \rem{t_{h + 1}} = 0$ and $\deg{f_1} \leq d$, the decoding was successful.
                    Checking the decoding by evaluating $f_1$ in $\Xi$:
                    \begin{align*}
                        f_1(\Xi) = (5, 12, 6, 0, 7, 1)
                    \end{align*}
                    Since the evaluation agrees with $\Gamma$ in all but two points, the decoding is indeed successful.
          \end{enumerate}

          \newpage

    \item \begin{enumerate}
              \item Let $M = \begin{bmatrix}
                            a & b \\ c & d
                        \end{bmatrix}$ be a $2\times2$-matrix. The determinant of $M$ is calculated as $\det{M} = ad - bc$.
                    Also, the inverse of $M$ is $\frac{1}{\det{M}}\begin{bmatrix}
                            d & -b \\ -c&  a
                        \end{bmatrix}$,  $\det{M} \neq 0$. This gives us for all $i = 0, 1, 2\dots$, $\det{Q_i} = -1$ and
                    \begin{align*}
                        \det{R_i} & = \det{Q_0}\cdot\det{Q_1}\cdot ... \cdot \det{Q_i} \\
                                  & = (-1)^i
                    \end{align*}
                    And furthermore,
                    \begin{align*}
                        R_i^{-1} & = \frac{1}{(-1)^i}\begin{bmatrix}
                                                         t_{i + 1} & -t_i \\ -s_{i + 1} & s_i
                                                     \end{bmatrix} \\
                                 & = (-1)^i\begin{bmatrix}
                                               t_{i + 1} & -t_i \\ -s_{i + 1} & s_i
                                           \end{bmatrix}           \\
                    \end{align*}
              \item \begin{align*}
                        R_i\begin{bmatrix}
                               r_0 \\ r_1
                           \end{bmatrix}                      & = \begin{bmatrix}
                                                                      r_i \\ r_{i + 1}
                                                                  \end{bmatrix}                                \\
                        \begin{bmatrix}
                            r_0 \\ r_1
                        \end{bmatrix}                         & = R^{-1}_i\begin{bmatrix}
                                                                              r_i \\ r_{i + 1}
                                                                          \end{bmatrix}                        \\
                        \intertext{Set $i = l$, and by definition $r_{l + 1} = 0$}
                                                                & = R^{-1}_l\begin{bmatrix}
                                                                                r_l \\ 0
                                                                            \end{bmatrix}                       \\
                                                                & = (-1)^l\begin{bmatrix}
                                                                              t_{l + 1} & -t_l \\ -s_{l + 1} & s_l
                                                                          \end{bmatrix}\begin{bmatrix}
                                                                                           r_l \\ 0
                                                                                       \end{bmatrix}    \\
                                                                & = (-1)^l\begin{bmatrix}
                                                                              r_l t_{l + 1} \\ r_l(-s_{l + 1})
                                                                          \end{bmatrix}        \\
                                                                & = \begin{bmatrix}
                                                                        (-1)^lr_l t_{l + 1} \\ (-1)^{l+1}r_ls_{l + 1}
                                                                    \end{bmatrix}
                        \intertext{Divide both sides by $r_l$:}
                        \begin{bmatrix}
                            (-1)^l t_{l + 1} \\ (-1)^{l+1}s_{l + 1}
                        \end{bmatrix} & = \begin{bmatrix}
                                              \frac{r_0}{r_l} \\ \frac{r_1}{r_l}
                                          \end{bmatrix}
                    \end{align*}
                    Thus the claim holds.
          \end{enumerate}

          \newpage

    \item

          \newpage
\end{enumerate}
\end{document}
