\documentclass[11pt,a4paper]{article}
\input{auxiliary.tex}

%-----------------------------------------------
%-----------------------------------------------

\title{CS-E4500 Advanced Course in Algorithms}
\author{Aleksi Kääriäinen  \\
	Aalto University  \\
	}

\begin{document}

\date{\today}

\maketitle

\newpage

\section*{Week 3 exercices}

\begin{enumerate}
    \item
          \begin{enumerate}
              \item Let $\alpha= 22122.21201$ and $\beta = 22121.22001$ in base $B = 3$. The multiplication $\alpha \beta$ is:
                    \begin{center}
                        \begin{tabular}{crl}
                                    & 2212221201           & $\cdot 3^{-5}$  \\
                            $\cdot$ & 2212122001           & $\cdot 3^{-5}$  \\
                            \hline
                                    & 22022010111020220201 & $\cdot 3^{-10}$ \\
                        \end{tabular}
                    \end{center}
                    The result of $\alpha \beta$ is 2.2022010111020220201 $\cdot 3^{9}$.

              \item Let $\alpha = 145.2332632$ and $\beta = 1345053.103$ in base $B = 7$. Calculating the addition $\alpha + \beta$
                    by reducing it to an integer addition by padding the numbers so that the radix points align:
                    \begin{center}
                        \begin{tabular}{crl}
                                & 145     & .2332632 \\
                            $+$ & 1345053 & .103     \\
                            \hline
                                & 1345231 & .3362632 \\
                        \end{tabular}
                    \end{center}
                    The result of $\alpha + \beta$ in base 7 is 1.3452313362632 $\cdot 7^{6}$.
          \end{enumerate}
          \newpage

    \item Note that $\rev_n a(x) = x^n \cdot a(\frac{1}{x})$. It holds that:
          \begin{align*}
              a                        & = qb + r                                                                                                             \\
              a(x)                     & = q(x)b(x) + r(x)
              \intertext{Substituting $x \implies \frac{1}{x}$:}
              a(\frac{1}{x})           & = q(\frac{1}{x})b(\frac{1}{x}) + r(\frac{1}{x})                                                                      \\
              \intertext{Then multiplying both sides with $x^n$:}
              x^n \cdot a(\frac{1}{x}) & = x^n \cdot q(\frac{1}{x})b(\frac{1}{x}) + x^n \cdot r(\frac{1}{x})                                                  \\
              \rev_n a(x)              & = x^{n - m} \cdot q(\frac{1}{x}) \cdot x^m \cdot b(\frac{1}{x}) + x^{n - m + 1} \cdot x^{m - 1} \cdot r(\frac{1}{x}) \\
              \intertext{Note that $\deg b = m$ and $b$ is monic implies that $\deg q = n - m$ and $\deg r \leq m - 1$.
                  This shows that the terms on the right-hand side of the equation are actually the reversals of the original polynomials $q, b, r$:}
              \rev_n a(x)              & = \rev_{n - m} q(x) \cdot \rev_m b(x) + x^{n - m + 1} \cdot \rev_{m - 1} r(x)                                        \\
              \rev_n a                 & = (\rev_{n - m} q)(\rev_m b) + x^{n - m + 1} \rev_{m - 1} r                                                          \\
          \end{align*}
          \newpage

    \item Let $\alpha = sB^e \sum_{i = 0}^{d - 1} \alpha_i B^{-i}$ and $\tilde{\alpha} = \tilde{s}B^{\tilde{e}}\sum_{i = 0}^{\tilde{d} - 1}
              \tilde{\alpha}_i B^{-i}$, both $\in \mathbb{Q}_B$. Assume that $d \geq \tilde{d}$. Since $d \geq \tilde{d}$, $\tilde{\alpha}$ can be padded with trailing zeros so that:
          \begin{align*}
              \tilde{\alpha}_i =
              \begin{cases}
                  \tilde{\alpha}_i, & i \leq \tilde{d}  \\
                  0 \hspace{1cm}    & \text{otherwise.}
              \end{cases}
              \intertext{Now we can write:}
              \tilde{\alpha} = \tilde{s}B^{\tilde{e}}\sum_{i = 0}^{d - 1}\tilde{\alpha}_i B^{-i}
          \end{align*}
          To be able to perform addition on two radix-point numbers, we first have to align the radix points to the same place:
          \begin{align*}
              \alpha         & = sB^e \sum_{i = 0}^{d - 1} \alpha_i B^{-i}                                           \\
              \tilde{\alpha} & = \tilde{s}B^{\tilde{e}}\sum_{i = 0}^{d - 1} \tilde{\alpha}_i B^{-i}                  \\
                             & = \tilde{s}B^{e} \cdot B^{\tilde{e} - e}\sum_{i = 0}^{d - 1} \tilde{\alpha}_i B^{-i}  \\
                             & = \tilde{s}B^{e} \sum_{i = 0}^{d - 1} \tilde{\alpha}_i B^{-i} \cdot B^{\tilde{e} - e} \\
                             & = \tilde{s}B^{e} \sum_{i = 0}^{d - 1} \tilde{\alpha}_i B^{\tilde{e} - e - i}
          \end{align*}
          Now that the radix points have been aligned, we can reduce the numbers to:
          \begin{align*}
              \alpha         & = sB^{e - d + 1} \sum_{i = 0}^{d - 1} \alpha_i B^{d - 1 - i}                                 \\
              \tilde{\alpha} & = \tilde{s}B^{e - d + 1} \sum_{i = 0}^{d - 1} \tilde{\alpha}_i B^{d - 1 + \tilde{e} - e - i}
          \end{align*}
          Notice that the sum term in $\alpha$ is always an integer. In $\tilde{\alpha}$ however, the sum term is an integer
          if and only if $\tilde{e} - e$ is non-negative. To ensure $\tilde{\alpha}$ to be an integer, we may pad both numbers with additional
          zeros $c$. We get:
          \begin{align*}
              \alpha         & = sB^{e - d + 1 - c} \sum_{i = 0}^{d - 1} \alpha_i B^{d - 1 - i + c}                                 \\
              \tilde{\alpha} & = \tilde{s}B^{e - d + 1 - c} \sum_{i = 0}^{d - 1} \tilde{\alpha}_i B^{d - 1 + \tilde{e} - e - i + c}
          \end{align*}
          We can now calculate the addition $\alpha + \tilde{\alpha}$:
          \begin{align*}
              \alpha + \tilde{\alpha} = sB^{e - d + 1 - c} \sum_{i = 0}^{d - 1} \alpha_i B^{d - 1 - i + c} +
              \tilde{s}B^{e - d + 1 - c} \sum_{i = 0}^{d - 1} \tilde{\alpha}_i B^{d - 1 + \tilde{e} - e - i + c}
          \end{align*}
          There are three cases depending on the values of $s$ and $\tilde{s}$:
          \begin{enumerate}
              \item $s, \tilde{s}$ are both 1, so the operation is addition. We get:
                    \begin{align*}
                          & B^{e - d + 1 - c} \sum_{i = 0}^{d - 1} \alpha_i B^{d - 1 - i + c} + \tilde{s}B^{e - d + 1 - c} \sum_{i = 0}^{d - 1} \tilde{\alpha}_i B^{d - 1 + \tilde{e} - e - i + c} \\
                        = & B^{e - d + 1 - c} \left( \sum_{i = 0}^{d - 1} \alpha_i B^{d - 1 - i + c} + \sum_{i = 0}^{d - 1} \tilde{\alpha}_i B^{d - 1 + \tilde{e} - e - i + c} \right)
                    \end{align*}
                    Notice that both sum terms in the parentheses are integers, and since it is known that integers are closed under addition, the result of the expression is a
                    radix-point representation of a number in base $B$. The number of digits $\hat{d} \leq \max{(e, \tilde{e})} + \max(d - e, \tilde{d} - \tilde{e}) + 1$ because of the potential carry in addition that results in an extra digit.
              \item $s \neq \tilde{s}$, so the operation is substraction.
                    \begin{align*}
                        B^{e - d + 1 - c} \left( \sum_{i = 0}^{d - 1} \alpha_i B^{d - 1 - i + c} - \sum_{i = 0}^{d - 1} \tilde{\alpha}_i B^{d - 1 + \tilde{e} - e - i + c} \right)
                    \end{align*}
                    Same as in (a), though the number of digits $\hat{d} \leq \max{(e, \tilde{e})} + \max(d - e, \tilde{d} - \tilde{e})$.
              \item $s, \tilde{s}$ are both -1, so the operation is negating the result of the addition. Since we established in (a) that $\mathbb{Q}_B$
                    is closed under addition, and it is known that $a, -a \in \mathbb{Q}_B$, the statement holds.
          \end{enumerate}
          \newpage

    \item

          \newpage
\end{enumerate}
\end{document}
