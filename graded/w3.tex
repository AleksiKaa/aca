\documentclass[11pt,a4paper]{article}
\input{auxiliary.tex}

%-----------------------------------------------
%-----------------------------------------------

\title{CS-E4500 Advanced Course in Algorithms}
\author{Aleksi Kääriäinen  \\
	Aalto University  \\
	}

\begin{document}

\date{\today}

\maketitle

\newpage

\section*{Week 3 exercices}

\begin{enumerate}
    \item
          \begin{enumerate}
              \item Let $\alpha= 22122.21201$ and $\beta = 22121.22001$ in base $B = 3$. The multiplication $\alpha \beta$ is:
                    \begin{center}
                        \begin{tabular}{crl}
                                    & 2212221201 & $\cdot 3^{-5}$      \\
                            $\cdot$ & 2212122001 & $\cdot 3^{-5}$     \\
                            \hline
                                    & 22022010111020220201 & $\cdot 3^{-10}$\\
                        \end{tabular}
                    \end{center}
                    The result of $\alpha \beta$ is 2.2022010111020220201 $\cdot 3^{9}$.

              \item Let $\alpha = 145.2332632$ and $\beta = 1345053.103$ in base $B = 7$. Calculating the addition $\alpha + \beta$
                    by reducing it to an integer addition by padding the numbers so that the radix points align:
                    \begin{center}
                        \begin{tabular}{crl}
                                & 145     & .2332632 \\
                            $+$ & 1345053 & .103     \\
                            \hline
                                & 1345231 & .3362632 \\
                        \end{tabular}
                    \end{center}
                    The result of $\alpha + \beta$ in base 7 is 1.3452313362632 $\cdot 7^{6}$.
          \end{enumerate}
          \newpage

    \item Note that $\rev_n a(x) = x^n \cdot a(\frac{1}{x})$. It holds that:
          \begin{align*}
              a                        & = qb + r                                                                                                             \\
              a(x)                     & = q(x)b(x) + r(x)
              \intertext{Substituting $x \implies \frac{1}{x}$:}
              a(\frac{1}{x})           & = q(\frac{1}{x})b(\frac{1}{x}) + r(\frac{1}{x})                                                                      \\
              \intertext{Then multiplying both sides with $x^n$:}
              x^n \cdot a(\frac{1}{x}) & = x^n \cdot q(\frac{1}{x})b(\frac{1}{x}) + x^n \cdot r(\frac{1}{x})                                                  \\
              \rev_n a(x)        & = x^{n - m} \cdot q(\frac{1}{x}) \cdot x^m \cdot b(\frac{1}{x}) + x^{n - m + 1} \cdot x^{m - 1} \cdot r(\frac{1}{x}) \\
              \intertext{Note that $\deg b = m$ and $b$ is monic implies that $\deg q = n - m$ and $\deg r \leq m - 1$.
                  This shows that the terms on the right-hand side of the equation are actually the reversals of the original polynomials $q, b, r$:}
              \rev_n a(x)        & = \rev_{n - m} q(x) \cdot \rev_m b(x) + x^{n - m + 1} \cdot \rev_{m - 1} r(x)                      \\
              \rev_n a           & = (\rev_{n - m} q)(\rev_m b) + x^{n - m + 1} \rev_{m - 1} r \\
          \end{align*}
          \newpage

    \item

          \newpage

    \item

          \newpage
\end{enumerate}
\end{document}
