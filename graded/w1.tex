\documentclass[11pt,a4paper]{article}
\usepackage{hyperref}
\usepackage[left=1.5in,right=1.5in,top=1.5in,bottom=1.5in]{geometry}
\usepackage{enumitem}
\usepackage{amssymb,amsmath,mathrsfs,amsthm}
\usepackage{parskip, graphicx, float, wrapfig, subcaption}


\setlength{\parindent}{0pt}
\setlength{\parskip}{3mm}

\pagestyle{empty}

%Title and formatting stuff
\usepackage{titlesec}
\titleformat{\chapter}
{\normalfont\Huge\color{aaltoBlue}}{\thechapter}{20pt}{\filleft\Huge}[\vskip 0.5ex\titlerule]
\usepackage{fancyhdr}
\pagestyle{fancy}
\fancyhead[L]{}
\newcommand{\mymarginpar}[1]{\marginpar{\it \small \raggedleft #1}}
\renewcommand*\thesection{\arabic{section}}

% Miscellaneous LaTeX macros
\newcommand{\heading}[1]{{\Large\color{blue!80!black}\textbf{#1}}\par}
\newcommand{\slemp}[1]{{\color{red}#1}}
\newcommand{\slred}[1]{{\color{red}#1}}
\newcommand{\slbf}[1]{{\color{blue}\textbf{#1}}}
\newcommand{\pair}[2]{\langle {#1}, {#2} \rangle}
\newcommand{\Abar}{\bar{A}}
\newcommand{\ALG}{\text{ALG}}
\newcommand{\approxA}{\alpha_{\cal A}}
\newcommand{\argmin}{\operatornamewithlimits{arg\ min}}
\newcommand{\assign}{\mbox{$\; \leftarrow \;$}}
\newcommand{\bbar}{\bar{b}}
\newcommand{\Bin}{\text{Bin}}
\newcommand{\calA}{{\cal A}}
\newcommand{\calC}{{\cal C}}
\newcommand{\calI}{{\cal I}}
\newcommand{\calM}{{\cal M}}
\newcommand{\calN}{{\cal N}}
\newcommand{\calS}{{\cal S}}
\newcommand{\chat}{\hat{c}}
\newcommand{\chatopt}{\hat{c}^*}
\newcommand{\copt}{c^*}
\newcommand{\Cov}{\text{Cov}}
\newcommand{\DC}{\text{DC}}
\newcommand{\delC}{\partial C}
\newcommand{\delS}{\partial S}
\newcommand{\delT}{\partial T}
\newcommand{\diag}{\textit{diag}}
\newcommand{\dtv}{d_{\text{V}}}
\newcommand{\dom}{\mathcal{D}}
\newcommand{\E}{\text{E}}
\newcommand{\eps}{\varepsilon}
\newcommand{\false}{\text{false}}
\newcommand{\Gbar}{\bar{G}}
\newcommand{\Gnp}{{\cal G}(n,p)}
\newcommand{\Int}{\mathbf{I}}
\newcommand{\lambdamax}{\lambda_{\text{max}}}
\newcommand{\LPI}{$\text{LP}_{\text{I}}$}
\newcommand{\LPII}{$\text{LP}_{\text{II}}$}
\newcommand{\NN}{\mathbb{N}}
\newcommand{\om}{\omega}
\newcommand{\Om}{\Omega}
\newcommand{\OPT}{\text{OPT}}
\newcommand{\OPTf}{\text{OPT}_f}
\newcommand{\phase}{\bar{\sigma}}
\newcommand{\phat}{\hat{p}}
\newcommand{\pt}{p^{(t)}}
\newcommand{\rbot}{r^{\bot}}
\newcommand{\rhat}{\hat{r}}
\newcommand{\RENT}{\text{RENT}}
\newcommand{\RR}{\mathbb{R}}
\newcommand{\RRmn}{\mathbb{R}^{m\times n}}
\newcommand{\RRn}{\mathbb{R}^n}
\newcommand{\RRnn}{\mathbb{R}^{n\times n}}
\newcommand{\sgn}{\text{sgn}}
\newcommand{\stot}{$s$-$t$}
\newcommand{\true}{\text{true}}
\newcommand{\st}{\text{s.t.}}
\newcommand{\vivi}{v_i^Tv_i}
\newcommand{\vivj}{v_i^Tv_j}
\newcommand{\Var}{\text{Var}}
\newcommand{\xbar}{\bar{x}}
\newcommand{\xopt}{x^*}
\newcommand{\ybar}{\bar{y}}
\newcommand{\yopt}{y^*}
\newcommand{\zbar}{\bar{z}}
\newcommand{\zopt}{z^*}
\newcommand{\ZIP}{Z^*_{\text{IP}}}
\newcommand{\ZLP}{Z^*_{\text{LP}}}
\newcommand{\ZZ}{\mathbb{Z}}

\newcommand{\tD}[1]{\textrm{\em\color{aaltoGreen}{#1}}}
\newcommand{\tB}[1]{\text{\color{aaltoBlack}{#1}}}
\newcommand{\Pt}{\ensuremath{\mathrm{P}}}
\newcommand{\NP}{\ensuremath{\mathrm{NP}}}
\newcommand{\bra}{\ensuremath{\langle}}
\newcommand{\ket}{\ensuremath{\rangle}}
\newcommand{\DFT}{\ensuremath{\operatorname{DFT}}}
\newcommand{\rev}{\ensuremath{\operatorname{rev}}}
\newcommand{\quo}{\ensuremath{\operatorname{quo}}}
\newcommand{\rem}{\ensuremath{\operatorname{rem}}}
\newcommand{\take}{\ensuremath{\operatorname{\upharpoonright}}}
\newcommand{\ord}{\ensuremath{\operatorname{ord}}}
\newtheorem{fact}{Fact}
\newcommand\Edit{\mathrm{Edit}}
\newcommand\backtrack{\stackrel{\text{back-track}}{\longrightarrow}}

\newcommand\N{\mathbb{N}} % Sorry did not see \NN etc earlier and now my \N  is everywhere :(
\newcommand\R{\mathbb{R}}
\newcommand\Z{\mathbb{Z}}

%--------- definition environment commands --

\theoremstyle{definition}
\newtheorem{definition}{Definition}[section]
\newtheorem{lemma}{Lemma}[section]
\newtheorem{corollary}{Corollary}[section]
\newtheorem{theorem}{Theorem}[section]
\newtheorem{example}{Example}[section]


%--------- STUFF FOR 2021 ITERATION ---------

\usepackage{bbm}
\usepackage{float}
\usepackage{tikz}
\newcommand{\ex}{\text{E}}
\newcommand{\pr}{\text{P}}
\allowdisplaybreaks

% question title
\newcommand{\qtitle}[1]{{\color{aaltoBlue}\textbf{#1}}}

%% algorithm environment
\usepackage[ruled, vlined]{algorithm2e}
\newcommand\mycommfont[1]{\footnotesize\ttfamily\textcolor{aaltoBlue}{#1}}
\SetCommentSty{mycommfont}
\SetFuncSty{text}
\SetKwProg{Fn}{def}{:}{}
\usepackage{ulem}

\newif\ifanswer
\newif\ifgraded
\newcommand{\InsertAnswer}[1]{\vspace{1mm}{\color{aaltoBlue}\ifanswer \textbf{Solution.}#1\fi}}

%an environment to automatically hide answers. Wrap \InsertAnswer with this
\newenvironment{answer}{}

\newcommand{\createheader}[1]{
{\color{aaltoBlue}{
\LARGE\bf CS-E3190 Principles of Algorithmic Techniques \\ [2mm]
{\it #1 \ifgraded -- Graded Exercise \else -- Tutorial Exercise \fi}

}}}


\newcommand{\rulebox}[1]{
  \ifgraded

    \vspace{-3mm}
    {\color{aaltoBlue} \rule{16cm}{1pt}}

    \centering{
      \vspace{2mm}
      \fbox{\begin{minipage}{14.5cm}
          \vspace{2mm} \small \normalfont
          {Please read the following \textbf{rules} very carefully.
            \begin{itemize}[itemsep=0.5mm, leftmargin=15pt]
              \item Do not consciously search for the solution on the internet.
              \item You are allowed to discuss the problems with your classmates but you should \textbf{write the solutions yourself}.
              \item Be aware that \textbf{if plagiarism is suspected}, you could be asked to have an interview with teaching staff.
              \item The teaching staff can assist with understanding the problem statements, but will \textbf{not be giving any hints} on how to solve the exercises.
              \item The use of generative AI tools of any kind is \textbf{not allowed}.
              \item In order to ease grading, we want the solution of each problem and subproblem to start on a \textbf{new page}. If this requirement is not met, \textbf{points will be deduced}.

            \end{itemize}
          }
          \vspace{0.5mm}
        \end{minipage}}}
  \else
    \vspace{-3mm}
    {\color{aaltoBlue} \rule{16cm}{1pt}}
    \vspace{-2mm}
  \fi
}

%-----------------------------------------------
%-----------------------------------------------

\begin{document}
\begin{enumerate}
	\item
	      \begin{enumerate}
		      \item $a = x + x^2$, \hspace{1cm} $b = 1 + x + x^3$, \hspace{1cm} $a, b \in \mathbb{Z}_2[x]$
		            \begin{align*}
			            a \cdot b & = (x + x^2) \cdot (1 + x + x^3)   \\
			                      & = x + x^2 + x^4 + x^2 + x^3 + x^5 \\
			                      & =x + x^3 + x^4 + x^5
		            \end{align*}
		      \item $a = 1 + x^2 + x^3 + x^4 + x^6$, \hspace{1cm} $b = 1 + x^3 + x^4$, \hspace{1cm} $a, b \in \mathbb{Z}_2[x]$
		            \begin{center}
			            \begin{tabular}{ c|c|c }

				            i & q         & r                           \\
				            \hline
				            0 & 0         & $1 + x^2 + x^3 + x^4 + x^6$ \\
				            1 & $x^2$     & $1 + x^3 + x^4 + x^5$       \\
				            2 & $x^2 + x$ & $1 + x + x^3$               \\
			            \end{tabular}
		            \end{center}
		            The quotient of $a / b$ is $x^2 + x$ and the remainder is $1 + x + x^3$.
	      \end{enumerate}

	      \newpage

	\item
	      \begin{enumerate}
		      \item Let  $f = 1234567$ and $g = 123$. Running the extended Euclidean algorithm on these inputs, we obtain:
		            \begin{center}
			            \begin{tabular}{l|l|l|l|l}
				              & r       & s   & t        & q     \\
				            \hline
				            0 & 1234567 & 1   & 0        & 0     \\
				            1 & 123     & 0   & 1        & 10037 \\
				            2 & 16      & 1   & -10037   & 7     \\
				            3 & 11      & -7  & 70260    & 1     \\
				            4 & 5       & 8   & -80297   & 2     \\
				            5 & 1       & -23 & 230854   & 5     \\
				            6 & 0       & 123 & -1234567 & 0     \\
			            \end{tabular}
		            \end{center}

		            The greatest common divisor of $f$ and $g$ is the last non-zero remainder $r_i$. Thus $gcd(f, g) = 1$.

		            TODO: find $g^{-1} \in \mathbb{Z}_f$

		      \item Let  $f = 1 + x + x^3 + x^4$ and $g = 1 + x^4$. Running the extended Euclidean algorithm on these inputs, we obtain:
		            \begin{center}
			            \begin{tabular}{l|l|l|l|l}
				            i & r                   & s         & t             & q   \\
				            \hline
				            0 & $x^4 + x^3 + x + 1$ & 1         & 0             & 0   \\
				            1 & $x^4 + 1$           & 0         & 1             & 1   \\
				            2 & $x^3 + x$           & 1         & 1             & $x$ \\
				            3 & $x^2 + 1$           & $x$       & $x + 1$       & $x$ \\
				            4 & 0                   & $x^2 + 1$ & $x^2 + x + 1$ & 0   \\
			            \end{tabular}
		            \end{center}
		            Like in the previous section, $gcd(f, g)$ is the final non-zero remainder $r_i$, thus it is $x^2 + 1$.
	      \end{enumerate}

	      \newpage

	\item The Lagrange polynomial $l_i$ for row $i$ of the Vandermonde matrix yields a polynomial of the form
	      $\prod^d_{_{j = 0}^{j \neq i}}\frac{x - \xi_j}{\xi_i - \xi_j} =  \sum_{k=0}^d
		      \lambda_{ik}x^k$. By arranging the coefficients $\lambda_{ik}$ of the polynomial in the form of a matrix, we get:
	      \begin{align*}
		      L = \begin{bmatrix}
			          \lambda_{00} & \lambda_{01} & \ldots & \lambda_{0d} \\
			          \lambda_{10} & \lambda_{11} & \ldots & \lambda_{1d} \\
			          \vdots       & \vdots       & \ddots & \vdots       \\
			          \lambda_{d0} & \lambda_{d1} & \ldots & \lambda_{dd}
		          \end{bmatrix}
	      \end{align*}
	      Now, computing the matrix multiplication
	      \begin{align*}
		      \Xi \cdot L & = \begin{bmatrix}
			                      \sum_{k= 0}^{d}\lambda_{k0}\xi_0^k & \sum_{k= 0}^{d}\lambda_{k1}\xi_0^k & \ldots & \sum_{k= 0}^{d}\lambda_{k2}\xi_0^k \\
			                      \sum_{k= 0}^{d}\lambda_{k0}\xi_1^k & \sum_{k= 0}^{d}\lambda_{k1}\xi_1^k & \ldots & \sum_{k= 0}^{d}\lambda_{k2}\xi_1^k \\
			                      \vdots                             & \vdots                             & \ddots & \vdots                             \\
			                      \sum_{k= 0}^{d}\lambda_{k0}\xi_d^k & \sum_{k= 0}^{d}\lambda_{k1}\xi_d^k & \ldots & \sum_{k= 0}^{d}\lambda_{k2}\xi_d^k \\
		                      \end{bmatrix} \\
		                  & = \begin{bmatrix}
			                      l_0(\xi_0) & l_1(\xi_0) & \ldots & l_d(\xi_0) \\
			                      l_0(\xi_1) & l_1(\xi_1) & \ldots & l_d(\xi_1) \\
			                      \vdots     & \vdots     & \ddots & \vdots     \\
			                      l_0(\xi_d) & l_1(\xi_d) & \ldots & l_d(\xi_d)
		                      \end{bmatrix}                                                                         \\
		      \intertext{Notice from the product form of the Lagrange polynomial that $l_i(\xi_j) = 0$, for any $\xi_j \neq \xi_i, \xi_j \in [0, d]$,
			      and $l_i(\xi_i) = 1$. Thus the multiplication results in:}
		                  & = \begin{bmatrix}
			                      1      & 0      & \ldots & 0      \\
			                      0      & 1      & \ldots & 0      \\
			                      \vdots & \vdots & \ddots & \vdots \\
			                      0      & 0      & \ldots & 1
		                      \end{bmatrix} = \mathbb{I}
	      \end{align*}
	      Since the result of the multiplication is the identity matrix, the Vandermonde matrix is invertible, and its inverse $\Xi^{-1} =
		      L$.
	      \newpage

	\item
	      \begin{enumerate}
		      \item We prove this claim with induction on $i$. According to the algorithm's definition, $R_0 = \begin{bmatrix}
				            1 & 0 \\ 0 & 1
			            \end{bmatrix}$. For the base case $i = 0$:
		            \begin{align*}
			            R_0\begin{bmatrix}
				               f \\
				               g
			               \end{bmatrix} & = \begin{bmatrix}
				                                 s_0 & t_0 \\
				                                 s_1 & t_1
			                                 \end{bmatrix} \begin{bmatrix}
				                                               f \\ g
			                                               \end{bmatrix}   \\
			                               & = \begin{bmatrix}
				                                   1 & 0 \\
				                                   0 & 1
			                                   \end{bmatrix} \begin{bmatrix}
				                                                 f \\ g
			                                                 \end{bmatrix} \\
			                               & = \begin{bmatrix}
				                                   f \\ g
			                                   \end{bmatrix}
		            \end{align*}
		            Assume that the statement holds for general $i$. Case $k = i + 1$:
		            \begin{align*}
			                                                                    & R_k\begin{bmatrix}
				                                                                         f                  \\ g                   \end{bmatrix}         \\
			                                                                    & = R_{i + 1}\begin{bmatrix}
				                                                                                 f                  \\ g                   \end{bmatrix} \\
			                                                                    & = Q_{i + 1}R_i\begin{bmatrix}
				                                                                                    f \\ g
			                                                                                    \end{bmatrix}                                       \\
			                                                                    & = Q_{i + 1}\begin{bmatrix}
				                                                                                 r_i \\ r_{i + 1}
			                                                                                 \end{bmatrix}                                         \\
			                                                                    & = \begin{bmatrix}
				                                                                        0 & 1 \\ 1 & -q_{i + 1}         \end{bmatrix} \begin{bmatrix}
				                                                                                                                      r_i \\ r_{i + 1}
			                                                                                                                      \end{bmatrix}    \\
			                                                                    & = \begin{bmatrix}
				                                                                        r_{i + 1} \\ r_i - q_{i + 1}r_{i + 1}
			                                                                        \end{bmatrix}
			            \intertext{The Extended Euclidean algorithm defines $r_{i + 2}$ as $r_{i} - q_{i + 1}r_{i + 1}$. Thus:}
			            R_{i + 1}\begin{bmatrix}
				                     f                  \\ g                   \end{bmatrix} & = \begin{bmatrix}
				                                                                                 r_{i + 1} \\ r_{i + 2}
			                                                                                 \end{bmatrix}
		            \end{align*}
		            We have concluded that the base case holds, and that the next step $i + 1$ holds if the general case $i$ holds,
		            thus the statement is true by mathematical induction.

		            \newpage

		      \item Let $R_i = \begin{bmatrix}
				            s_i & t_i \\ s_{i + 1} & t_{i + 1}
			            \end{bmatrix}$. We give a proof of this by induction on $i$.

		            Base case: $i = 0$ \\
		            This is clearly always true, since the algorithm initializes the values of $R_0$ as $\begin{bmatrix}
				            1 & 0 \\ 0 & 1
			            \end{bmatrix}$ in the beginning of the algorithm.

		            Induction step: Show that if $R_i$ holds for every $i \geq 0$, then $R_{i + 1}$ holds as well. It follows that:
		            \begin{align*}
			            R_{i + 1} & = Q_{i + 1}R_i                                       \\
			                      & = \begin{bmatrix}
				                          0 & 1          \\
				                          1 & -q_{i + 1}
			                          \end{bmatrix} \begin{bmatrix}
				                                        s_i       & t_i       \\
				                                        s_{i + 1} & t_{i + 1} \\
			                                        \end{bmatrix}                \\
			                      & = \begin{bmatrix}
				                          s_{i + 1}                & t_{i + 1}               \\
				                          s_i - q_{i + 1}s_{i + 1} & t_i -q_{i + 1}t_{i + 1}
			                          \end{bmatrix}
			            \intertext{The extended Euclidean algorithm defines $s_{i + 2} = s_i - q_{i + 1}s_{i + 1}$ and $t_{i + 2} = t_i - q_{i + 1}t_{i + 1}$.
			            Thus we can write $R_{i + 1}$ as:}
			            R_{i + 1} & = \begin{bmatrix}
				                          s_{i + 1} & t_{i + 1} \\
				                          s_{i + 2} & t_{i + 2}
			                          \end{bmatrix}
		            \end{align*}
		            This proves that if $R_i$ holds, then so does $R_{i + 1}$. With the base case and induction step proven to be true,
		            the statement is true by mathematical induction when $i = 0, 1, 2, \dots, l$.

		            \newpage
		      \item For case $i = l$:
		            \begin{align*}
			            R_l\begin{bmatrix}
				               f \\ g
			               \end{bmatrix}                  & = \begin{bmatrix}
				                                                  r_l \\ r_{l + 1}
			                                                  \end{bmatrix}                 \\
			            Q_lR_{l - 1}\begin{bmatrix}
				                        f \\ g
			                        \end{bmatrix}         & = \begin{bmatrix}
				                                                  r_l \\ 0
			                                                  \end{bmatrix}                  \\
			            Q_l^{-1}Q_lR_{l - 1}\begin{bmatrix}
				                                f \\ g
			                                \end{bmatrix} & = Q_l^{-1}\begin{bmatrix}
				                                                          r_l \\ 0
			                                                          \end{bmatrix}          \\
			            R_{l - 1}\begin{bmatrix}
				                     f \\ g
			                     \end{bmatrix}            & = \begin{bmatrix}
				                                                  q_l & 1 \\ 1 & 0
			                                                  \end{bmatrix} \begin{bmatrix}
				                                                                r_l \\ 0
			                                                                \end{bmatrix}    \\
			            R_{l - 1}\begin{bmatrix}
				                     f \\ g
			                     \end{bmatrix}            & = \begin{bmatrix}
				                                                  q_lr_l \\ r_l
			                                                  \end{bmatrix} = \begin{bmatrix}
				                                                                  r_{l - 1} \\ r_l
			                                                                  \end{bmatrix}
		            \end{align*}
		            This computation shows that by multiplying both sides with the inverse of $Q_i$ allows traversing
		            the algorithm backwards, producing the previous values of $r$. This procedure can be repeated until it yields the input values $f, g$.
		            Additionally, the computation shows that the values of $r_i$ and $r_{i + 1}$ can be written in a form where both have the coefficient
		            $r_l$ as the only common part, meaning that $r_l$ is the greatest common divisor of every $r_i, r_{i + 1}$

		            \newpage

		      \item Insert the statement of (b) in to (a). This gives us:
		            \begin{align*}
			            R_i\begin{bmatrix}
				               f \\ g
			               \end{bmatrix} =
			            \begin{bmatrix}
				            s_i       & t_i       \\
				            s_{i + 1} & t_{i + 1}
			            \end{bmatrix} \begin{bmatrix}
				                          f \\ g
			                          \end{bmatrix} = & \begin{bmatrix}
				                                            s_if + t_ig \\
				                                            s_{i + 1}f + t_{i + 1}g
			                                            \end{bmatrix} = \begin{bmatrix}
				                                                            r_i \\
				                                                            r_{i + 1}
			                                                            \end{bmatrix}
		            \end{align*}
		            This proves that $s_if + t_ig = r_i$.
	      \end{enumerate}
\end{enumerate}
\end{document}