\documentclass[11pt,a4paper]{article}
\input{auxiliary.tex}

%-----------------------------------------------
%-----------------------------------------------

\title{CS-E4500 Advanced Course in Algorithms}
\author{Aleksi Kääriäinen  \\
	Aalto University  \\
	}

\begin{document}

\date{\today}

\maketitle

\newpage

\section*{Week 2 exercices}

\begin{enumerate}
    \item \begin{enumerate}
              \item Let $\omega = 4$ be a primitive root of order $n = 6$ in $\mathbb{Z}_{13}^6$.
                    \begin{align*}
                          & \text{DFT}_\omega(f)                  \\ = & \begin{bmatrix}
                            f(\omega^0) & f(\omega^1) & f(\omega^2) & f(\omega^3) & f(\omega^4) & f(\omega^5)
                        \end{bmatrix}^T \\
                        = & \begin{bmatrix}
                                3 & 21 & 273 & 4161 & 65793 & 1049601
                            \end{bmatrix}^T \mod{13} \\
                        = & \begin{bmatrix}
                                3 & 8 & 0 & 1 & 0 & 7
                            \end{bmatrix}^T
                    \end{align*}
                    Similarly, for $g$:
                    \begin{align*}
                          & \text{DFT}_\omega(g)                                 \\ = & \begin{bmatrix}
                            g(\omega^0) & g(\omega^1) & g(\omega^2) & g(\omega^3) & g(\omega^4) & g(\omega^5)
                        \end{bmatrix}^T \\
                        = & \begin{bmatrix}
                                14 & 770 & 49154 & 3145730 & 201326594 & 12884901890
                            \end{bmatrix}^T \mod{13} \\
                        = & \begin{bmatrix}
                                1 & 3 & 1 & 3 & 1 & 3
                            \end{bmatrix}^T
                    \end{align*}
              \item \begin{align*}
                          & \text{DFT}_\omega(f) \cdot \text{DFT}_\omega(g)                       \\
                        = & \begin{bmatrix}
                                3 \cdot 1 & 8 \cdot 3 & 0 \cdot 1 & 1 \cdot 3 & 0 \cdot 1 & 7 \cdot 3
                            \end{bmatrix}^T \mod{13} \\
                        = & \begin{bmatrix}
                                3 & 11 & 0 & 3 & 0 & 8
                            \end{bmatrix}^T
                    \end{align*}
              \item Let $\omega^{-1} = 10 \in \mathbb{Z}_{13}$ and $n^{-1} = 11 \in \mathbb{Z}_{13}$. Let $h$ denote the polynomial
                    where the coefficients of $x$ in increasing order are the entries of
                    $\text{DFT}_{\omega^{-1}}(\text{DFT}_\omega(f) \cdot \text{DFT}_\omega(g))$. Thus $h(x) = 3 + 11x + 3x^3 + 8x^5$
                    \begin{align*}
                          & \frac{1}{n}\text{DFT}_{\omega^{-1}}(\text{DFT}_\omega(f) \cdot \text{DFT}_\omega(g)) \\
                        = & 11 \cdot \text{DFT}_{10}(\text{DFT}_\omega(f) \cdot \text{DFT}_\omega(g))            \\
                        = & 11 \cdot \begin{bmatrix}
                                         h(10^0) & h(10^1) & h(10^2) & h(10^3) & h(10^4) & h(10^5)
                                     \end{bmatrix}^T \mod{13}                   \\
                        = & \begin{bmatrix}
                                2 & 2 & 2 & 12 & 12 & 12
                            \end{bmatrix}
                    \end{align*}
                    Thus the polynomial $fg(x) = 2 + 2x + 2x^2 + 12x^3 + 12x^4 + 12x^5$.
          \end{enumerate}

          \newpage

    \item At any $s$, $s \in [n]$:
          \begin{align*}
              \hat{f}_{s} & = V_{\omega, s}^{n \times n} \cdot f  \\
                          & = \sum_{t = 0}^{n - 1} \omega^{st}f_t
          \end{align*}
          Note that $n$ is composite, i.e. $n = ab$.
          By re-indexing $t$ with base $(a, b)$ and $s$ with base $(b, a)$, we get:
          \begin{align*}
              t                           & = bk + l, \hspace{2cm} k \in [a], l \in [b]                                                                   \\
              s                           & = ai + j, \hspace{2cm} i \in [b], j \in [a]                                                                   \\
              \intertext{Insert these values to $s, t$:}
              \hat{f}^{b \times a}_{i, j} & = \sum_{l = 0}^{b - 1} \sum_{k = 0}^{a - 1} \omega^{(ai + j)(bk + l)} f_{k, l}^{a \times b}                   \\
                                          & = \sum_{l \in [b]} \sum_{k \in [a]} \omega^{abik + bjk + ail + jl} f_{k, l}^{a \times b}                      \\
                                          & = \sum_{l \in [b]} \sum_{k \in [a]} \omega^{abik} \omega^{bjk} \omega^{ail} \omega^{jl} f_{k, l}^{a \times b}
          \end{align*}
          Since $\omega^{ab} = 1$, so is $\omega^{abik}$. Thus the expression can be reduced to:
          \begin{align*}
              \hat{f}_{i, j}^{b \times a} & = \sum_{l \in [b]} \sum_{k \in [a]} \omega^{bjk}\omega^{ail}\omega^{jl}  f_{k, l}^{a \times b}       \\
                                          & = \sum_{l \in [b]} \omega^{ail}\omega^{jl} \sum_{k \in [a]} \omega^{bjk} f_{k, l}^{a \times b}       \\
                                          & = \sum_{l \in [b]} \omega^{a^{il}}\omega^{jl} \sum_{k \in [a]} \omega^{b^{jk}} f_{k, l}^{a \times b}
          \end{align*}
          Since both $j, k \in [a]$, $ \sum_{k \in [a]}\omega^{b^{jk}}$ can be viewed as the Vandermonde matrix $V_{\omega^b}^{a \times a}$.
          \begin{align*}
              \hat{f}_{i, j}^{b \times a} & = \sum_{l \in [b]} \omega^{a^{il}}\omega^{jl} V_{\omega^b}^{a \times a} f_{k, l}^{a \times b}
          \end{align*}
          Note that $V_{\omega^b}^{a \times a} f_{k, l}^{a \times b}$ is a array of shape $a \times b$. The term $w^{jl}$ can be viewed as a
          Vandermonde matrix of shape $a \times b$ with base $\omega$. The product is thus the Hadamard product of the matrices.
          \begin{align*}
              \hat{f}_{i, j}^{b \times a} & = \sum_{l \in [b]} \omega^{a^{il}} V_\omega^{a \times b} \odot V_{\omega^b}^{a \times a} f_{k, l}^{a \times b}
          \end{align*}
          The remaining sum over $b$ takes the $a \times b$ array, and outputs the array obtained by taking the $a$-point DFT at $\omega^a$
          for each of the rows $l$ in the matrix. Thus:
          \begin{align*}
              \hat{f}_{i, j}^{b \times a} & = ((V_\omega^{a \times b} \odot V_{\omega^b}^{a \times a} f_{k, l}^{a \times b})V_{\omega^a}^{b \times b})^T
          \end{align*}
          The transpose is needed because the terms in the parenthesis result in an array of shape $a \times b$.
          Now, applying lemma 5 to the equation yields:
          \begin{align*}
              \hat{f}_{i, j}^{b \times a} & = ((\text{diag} V_w^{a \times b} V_{\omega^b}^{a \times a} f_{k, l}^{a \times b})V_{\omega^a}^{b \times b})^T
          \end{align*}
          Next, by lemma 4.2 $V_{\omega^b}^{a \times a} f_{k, l}^{a \times b} = ((V_{\omega^b} \otimes I_b)f_{k, l})^{a \times b}$.
          \begin{align*}
              \hat{f}_{i, j}^{b \times a} & = ((\text{diag} V_w^{a \times b} ((V_{\omega^b}\otimes I_b) f_{k, l})^{a \times b})V_{\omega^a}^{b \times b})^T             \\
                                          & = ((\text{diag} V_w^{a \times b} (V_{\omega^b}^{a \times b} \otimes I_b) f_{k, l}^{a \times b})V_{\omega^a}^{b \times b})^T
          \end{align*}
          Applying lemma 4.2 again to the right-hand side gives us:
          \begin{align*}
               & = ((I_a \otimes V_{\omega^a}^{b \times b})((\text{diag} V_w^{a \times b} (V_{\omega^b}^{a \times b} \otimes I_b) f_{k, l}^{a \times b}))^T)^T \\
               & = (I_a \otimes V_{\omega^a}^{b \times b})((\text{diag} V_w^{a \times b} (V_{\omega^b}^{a \times b} \otimes I_b) f_{k, l}^{a \times b}))       \\
               & = (I_a \otimes V_{\omega^a}^{b \times b})(\text{diag} V_w^{a \times b}) (V_{\omega^b}^{a \times b} \otimes I_b) f_{k, l}^{a \times b}
          \end{align*}
          Since the vector $f$ is arbitrary, the equation is true even when $f^{a \times b}$ does not alter the values of the right hand side. Thus:
          \begin{align*}
                & (I_a \otimes V_{\omega^a}^{b \times b})(\text{diag} V_w^{a \times b}) (V_{\omega^b}^{a \times b} \otimes I_b) f_{k, l}^{a \times b} \\
              = & (I_a \otimes V_{\omega^a}^{b \times b})(\text{diag} V_w^{a \times b}) (V_{\omega^b}^{a \times b} \otimes I_b)
          \end{align*}
          Finally, taking the base transposition matrix of the above yields us:
          \begin{align*}
            \hat{F}^{ab \times ab} & = \prod_{a \iff b}(I_a \otimes V_{\omega^a}^{b \times b})(\text{diag} V_w^{a \times b}) (V_{\omega^b}^{a \times b} \otimes I_b) \\
            \hat{F}^{n \times n} & = \prod_{a \iff b}(I_a \otimes V_{\omega^a}^{b \times b})(\text{diag} V_w^{a \times b}) (V_{\omega^b}^{a \times b} \otimes I_b)
          \end{align*}
          Which equals the Vandermonde matrix:
          \begin{align*}            
            V^{n \times n}_\omega & = \prod_{a \iff b}(I_a \otimes V_{\omega^a}^{b \times b})(\text{diag} V_w^{a \times b}) (V_{\omega^b}^{a \times b} \otimes I_b)
          \end{align*}


          \newpage
    \item

          \newpage

    \item

          \newpage
\end{enumerate}
\end{document}
