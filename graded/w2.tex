\documentclass[11pt,a4paper]{article}
\input{auxiliary.tex}

%-----------------------------------------------
%-----------------------------------------------

\title{CS-E4500 Advanced Course in Algorithms}
\author{Aleksi Kääriäinen  \\
	Aalto University  \\
	}

\begin{document}

\date{\today}

\maketitle

\newpage

\section*{Week 2 exercices}

\begin{enumerate}
    \item \begin{enumerate}
              \item Let $\omega = 4$ be a primitive root of order $n = 6$ in $\mathbb{Z}_{13}^6$.
                    \begin{align*}
                        & \text{DFT}_\omega(f)                      \\ = & \begin{bmatrix}
                            f(\omega^0) & f(\omega^1) & f(\omega^2) & f(\omega^3) & f(\omega^4) & f(\omega^5)
                        \end{bmatrix}^T \\
                        = & \begin{bmatrix}
                                3 & 21 & 273 & 4161 & 65793 & 1049601
                            \end{bmatrix}^T \mod{13} \\
                        = & \begin{bmatrix}
                                3 & 8 & 0 & 1 & 0 & 7
                            \end{bmatrix}^T
                    \end{align*}
                    Similarly, for $g$:
                    \begin{align*}
                          & \text{DFT}_\omega(g)                                 \\ = & \begin{bmatrix}
                            g(\omega^0) & g(\omega^1) & g(\omega^2) & g(\omega^3) & g(\omega^4) & g(\omega^5)
                        \end{bmatrix}^T \\
                        = & \begin{bmatrix}
                                14 & 770 & 49154 & 3145730 & 201326594 & 12884901890
                            \end{bmatrix}^T \mod{13} \\
                        = & \begin{bmatrix}
                                1 & 3 & 1 & 3 & 1 & 3
                            \end{bmatrix}^T
                    \end{align*}
              \item \begin{align*}
                          & \text{DFT}_\omega(f) \cdot \text{DFT}_\omega(g)                       \\
                        = & \begin{bmatrix}
                                3 \cdot 1 & 8 \cdot 3 & 0 \cdot 1 & 1 \cdot 3 & 0 \cdot 1 & 7 \cdot 3
                            \end{bmatrix}^T \mod{13} \\
                        = & \begin{bmatrix}
                                3 & 11 & 0 & 3 & 0 & 8
                            \end{bmatrix}^T
                    \end{align*}
              \item Let $\omega^{-1} = 10 \in \mathbb{Z}_{13}$ and $n^{-1} = 11 \in \mathbb{Z}_{13}$. Let $h$ denote the polynomial
                    where the coefficients of $x$ in increasing order are the entries of
                    $\text{DFT}_{\omega^{-1}}(\text{DFT}_\omega(f) \cdot \text{DFT}_\omega(g))$. Thus $h(x) = 3 + 11x + 3x^3 + 8x^5$
                    \begin{align*}
                          & \frac{1}{n}\text{DFT}_{\omega^{-1}}(\text{DFT}_\omega(f) \cdot \text{DFT}_\omega(g)) \\
                        = & 11 \cdot \text{DFT}_{10}(\text{DFT}_\omega(f) \cdot \text{DFT}_\omega(g))            \\
                        = & 11 \cdot \begin{bmatrix}
                                         h(10^0) & h(10^1) & h(10^2) & h(10^3) & h(10^4) & h(10^5)
                                     \end{bmatrix}^T \mod{13}                   \\
                        = & \begin{bmatrix}
                                2 & 2 & 2 & 12 & 12 & 12
                            \end{bmatrix}
                    \end{align*}
                    Thus the polynomial $fg(x) = 2 + 2x + 2x^2 + 12x^3 + 12x^4 + 12x^5$.
          \end{enumerate}

          \newpage

    \item

          \newpage

    \item

          \newpage

    \item

          \newpage
\end{enumerate}
\end{document}
