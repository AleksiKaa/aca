\documentclass[11pt,a4paper]{article}
\usepackage{hyperref}
\usepackage[left=1.5in,right=1.5in,top=1.5in,bottom=1.5in]{geometry}
\usepackage{enumitem}
\usepackage{amssymb,amsmath,mathrsfs,amsthm}
\usepackage{parskip, graphicx, float, wrapfig, subcaption}


\setlength{\parindent}{0pt}
\setlength{\parskip}{3mm}

\pagestyle{empty}

%Title and formatting stuff
\usepackage{titlesec}
\titleformat{\chapter}
{\normalfont\Huge\color{aaltoBlue}}{\thechapter}{20pt}{\filleft\Huge}[\vskip 0.5ex\titlerule]
\usepackage{fancyhdr}
\pagestyle{fancy}
\fancyhead[L]{}
\newcommand{\mymarginpar}[1]{\marginpar{\it \small \raggedleft #1}}
\renewcommand*\thesection{\arabic{section}}

% Miscellaneous LaTeX macros
\newcommand{\heading}[1]{{\Large\color{blue!80!black}\textbf{#1}}\par}
\newcommand{\slemp}[1]{{\color{red}#1}}
\newcommand{\slred}[1]{{\color{red}#1}}
\newcommand{\slbf}[1]{{\color{blue}\textbf{#1}}}
\newcommand{\pair}[2]{\langle {#1}, {#2} \rangle}
\newcommand{\Abar}{\bar{A}}
\newcommand{\ALG}{\text{ALG}}
\newcommand{\approxA}{\alpha_{\cal A}}
\newcommand{\argmin}{\operatornamewithlimits{arg\ min}}
\newcommand{\assign}{\mbox{$\; \leftarrow \;$}}
\newcommand{\bbar}{\bar{b}}
\newcommand{\Bin}{\text{Bin}}
\newcommand{\calA}{{\cal A}}
\newcommand{\calC}{{\cal C}}
\newcommand{\calI}{{\cal I}}
\newcommand{\calM}{{\cal M}}
\newcommand{\calN}{{\cal N}}
\newcommand{\calS}{{\cal S}}
\newcommand{\chat}{\hat{c}}
\newcommand{\chatopt}{\hat{c}^*}
\newcommand{\copt}{c^*}
\newcommand{\Cov}{\text{Cov}}
\newcommand{\DC}{\text{DC}}
\newcommand{\delC}{\partial C}
\newcommand{\delS}{\partial S}
\newcommand{\delT}{\partial T}
\newcommand{\diag}{\textit{diag}}
\newcommand{\dtv}{d_{\text{V}}}
\newcommand{\dom}{\mathcal{D}}
\newcommand{\E}{\text{E}}
\newcommand{\eps}{\varepsilon}
\newcommand{\false}{\text{false}}
\newcommand{\Gbar}{\bar{G}}
\newcommand{\Gnp}{{\cal G}(n,p)}
\newcommand{\Int}{\mathbf{I}}
\newcommand{\lambdamax}{\lambda_{\text{max}}}
\newcommand{\LPI}{$\text{LP}_{\text{I}}$}
\newcommand{\LPII}{$\text{LP}_{\text{II}}$}
\newcommand{\NN}{\mathbb{N}}
\newcommand{\om}{\omega}
\newcommand{\Om}{\Omega}
\newcommand{\OPT}{\text{OPT}}
\newcommand{\OPTf}{\text{OPT}_f}
\newcommand{\phase}{\bar{\sigma}}
\newcommand{\phat}{\hat{p}}
\newcommand{\pt}{p^{(t)}}
\newcommand{\rbot}{r^{\bot}}
\newcommand{\rhat}{\hat{r}}
\newcommand{\RENT}{\text{RENT}}
\newcommand{\RR}{\mathbb{R}}
\newcommand{\RRmn}{\mathbb{R}^{m\times n}}
\newcommand{\RRn}{\mathbb{R}^n}
\newcommand{\RRnn}{\mathbb{R}^{n\times n}}
\newcommand{\sgn}{\text{sgn}}
\newcommand{\stot}{$s$-$t$}
\newcommand{\true}{\text{true}}
\newcommand{\st}{\text{s.t.}}
\newcommand{\vivi}{v_i^Tv_i}
\newcommand{\vivj}{v_i^Tv_j}
\newcommand{\Var}{\text{Var}}
\newcommand{\xbar}{\bar{x}}
\newcommand{\xopt}{x^*}
\newcommand{\ybar}{\bar{y}}
\newcommand{\yopt}{y^*}
\newcommand{\zbar}{\bar{z}}
\newcommand{\zopt}{z^*}
\newcommand{\ZIP}{Z^*_{\text{IP}}}
\newcommand{\ZLP}{Z^*_{\text{LP}}}
\newcommand{\ZZ}{\mathbb{Z}}

\newcommand{\tD}[1]{\textrm{\em\color{aaltoGreen}{#1}}}
\newcommand{\tB}[1]{\text{\color{aaltoBlack}{#1}}}
\newcommand{\Pt}{\ensuremath{\mathrm{P}}}
\newcommand{\NP}{\ensuremath{\mathrm{NP}}}
\newcommand{\bra}{\ensuremath{\langle}}
\newcommand{\ket}{\ensuremath{\rangle}}
\newcommand{\DFT}{\ensuremath{\operatorname{DFT}}}
\newcommand{\rev}{\ensuremath{\operatorname{rev}}}
\newcommand{\quo}{\ensuremath{\operatorname{quo}}}
\newcommand{\rem}{\ensuremath{\operatorname{rem}}}
\newcommand{\take}{\ensuremath{\operatorname{\upharpoonright}}}
\newcommand{\ord}{\ensuremath{\operatorname{ord}}}
\newtheorem{fact}{Fact}
\newcommand\Edit{\mathrm{Edit}}
\newcommand\backtrack{\stackrel{\text{back-track}}{\longrightarrow}}

\newcommand\N{\mathbb{N}} % Sorry did not see \NN etc earlier and now my \N  is everywhere :(
\newcommand\R{\mathbb{R}}
\newcommand\Z{\mathbb{Z}}

%--------- definition environment commands --

\theoremstyle{definition}
\newtheorem{definition}{Definition}[section]
\newtheorem{lemma}{Lemma}[section]
\newtheorem{corollary}{Corollary}[section]
\newtheorem{theorem}{Theorem}[section]
\newtheorem{example}{Example}[section]


%--------- STUFF FOR 2021 ITERATION ---------

\usepackage{bbm}
\usepackage{float}
\usepackage{tikz}
\newcommand{\ex}{\text{E}}
\newcommand{\pr}{\text{P}}
\allowdisplaybreaks

% question title
\newcommand{\qtitle}[1]{{\color{aaltoBlue}\textbf{#1}}}

%% algorithm environment
\usepackage[ruled, vlined]{algorithm2e}
\newcommand\mycommfont[1]{\footnotesize\ttfamily\textcolor{aaltoBlue}{#1}}
\SetCommentSty{mycommfont}
\SetFuncSty{text}
\SetKwProg{Fn}{def}{:}{}
\usepackage{ulem}

\newif\ifanswer
\newif\ifgraded
\newcommand{\InsertAnswer}[1]{\vspace{1mm}{\color{aaltoBlue}\ifanswer \textbf{Solution.}#1\fi}}

%an environment to automatically hide answers. Wrap \InsertAnswer with this
\newenvironment{answer}{}

\newcommand{\createheader}[1]{
{\color{aaltoBlue}{
\LARGE\bf CS-E3190 Principles of Algorithmic Techniques \\ [2mm]
{\it #1 \ifgraded -- Graded Exercise \else -- Tutorial Exercise \fi}

}}}


\newcommand{\rulebox}[1]{
  \ifgraded

    \vspace{-3mm}
    {\color{aaltoBlue} \rule{16cm}{1pt}}

    \centering{
      \vspace{2mm}
      \fbox{\begin{minipage}{14.5cm}
          \vspace{2mm} \small \normalfont
          {Please read the following \textbf{rules} very carefully.
            \begin{itemize}[itemsep=0.5mm, leftmargin=15pt]
              \item Do not consciously search for the solution on the internet.
              \item You are allowed to discuss the problems with your classmates but you should \textbf{write the solutions yourself}.
              \item Be aware that \textbf{if plagiarism is suspected}, you could be asked to have an interview with teaching staff.
              \item The teaching staff can assist with understanding the problem statements, but will \textbf{not be giving any hints} on how to solve the exercises.
              \item The use of generative AI tools of any kind is \textbf{not allowed}.
              \item In order to ease grading, we want the solution of each problem and subproblem to start on a \textbf{new page}. If this requirement is not met, \textbf{points will be deduced}.

            \end{itemize}
          }
          \vspace{0.5mm}
        \end{minipage}}}
  \else
    \vspace{-3mm}
    {\color{aaltoBlue} \rule{16cm}{1pt}}
    \vspace{-2mm}
  \fi
}

%-----------------------------------------------
%-----------------------------------------------

\title{CS-E4500 Advanced Course in Algorithms}
\author{Aleksi Kääriäinen  \\
	Aalto University  \\
	}

\begin{document}

\date{\today}

\maketitle

\newpage

\section*{Week 2 exercices}

\begin{enumerate}
    \item \begin{enumerate}
              \item Let $\omega = 4$ be a primitive root of order $n = 6$ in $\mathbb{Z}_{13}^6$.
                    \begin{align*}
                          & \text{DFT}_\omega(f)                  \\ = & \begin{bmatrix}
                            f(\omega^0) & f(\omega^1) & f(\omega^2) & f(\omega^3) & f(\omega^4) & f(\omega^5)
                        \end{bmatrix}^T \\
                        = & \begin{bmatrix}
                                3 & 21 & 273 & 4161 & 65793 & 1049601
                            \end{bmatrix}^T \mod{13} \\
                        = & \begin{bmatrix}
                                3 & 8 & 0 & 1 & 0 & 7
                            \end{bmatrix}^T
                    \end{align*}
                    Similarly, for $g$:
                    \begin{align*}
                          & \text{DFT}_\omega(g)                                 \\ = & \begin{bmatrix}
                            g(\omega^0) & g(\omega^1) & g(\omega^2) & g(\omega^3) & g(\omega^4) & g(\omega^5)
                        \end{bmatrix}^T \\
                        = & \begin{bmatrix}
                                14 & 770 & 49154 & 3145730 & 201326594 & 12884901890
                            \end{bmatrix}^T \mod{13} \\
                        = & \begin{bmatrix}
                                1 & 3 & 1 & 3 & 1 & 3
                            \end{bmatrix}^T
                    \end{align*}
              \item \begin{align*}
                          & \text{DFT}_\omega(f) \cdot \text{DFT}_\omega(g)                       \\
                        = & \begin{bmatrix}
                                3 \cdot 1 & 8 \cdot 3 & 0 \cdot 1 & 1 \cdot 3 & 0 \cdot 1 & 7 \cdot 3
                            \end{bmatrix}^T \mod{13} \\
                        = & \begin{bmatrix}
                                3 & 11 & 0 & 3 & 0 & 8
                            \end{bmatrix}^T
                    \end{align*}
              \item Let $\omega^{-1} = 10 \in \mathbb{Z}_{13}$ and $n^{-1} = 11 \in \mathbb{Z}_{13}$. Let $h$ denote the polynomial
                    where the coefficients of $x$ in increasing order are the entries of
                    $\text{DFT}_{\omega^{-1}}(\text{DFT}_\omega(f) \cdot \text{DFT}_\omega(g))$. Thus $h(x) = 3 + 11x + 3x^3 + 8x^5$
                    \begin{align*}
                          & \frac{1}{n}\text{DFT}_{\omega^{-1}}(\text{DFT}_\omega(f) \cdot \text{DFT}_\omega(g)) \\
                        = & 11 \cdot \text{DFT}_{10}(\text{DFT}_\omega(f) \cdot \text{DFT}_\omega(g))            \\
                        = & 11 \cdot \begin{bmatrix}
                                         h(10^0) & h(10^1) & h(10^2) & h(10^3) & h(10^4) & h(10^5)
                                     \end{bmatrix}^T \mod{13}                   \\
                        = & \begin{bmatrix}
                                2 & 2 & 2 & 12 & 12 & 12
                            \end{bmatrix}
                    \end{align*}
                    Thus the polynomial $fg(x) = 2 + 2x + 2x^2 + 12x^3 + 12x^4 + 12x^5$.
          \end{enumerate}

          \newpage

    \item At any $s$, $s \in [n]$:
          \begin{align*}
              \hat{f}_{s} & = V_{\omega, s}^{n \times n} \cdot f  \\
                          & = \sum_{t = 0}^{n - 1} \omega^{st}f_t
          \end{align*}
          Note that $n$ is composite, i.e. $n = ab$.
          By re-indexing $t$ with base $(a, b)$ and $s$ with base $(b, a)$, we get:
          \begin{align*}
              t                           & = bk + l, \hspace{2cm} k \in [a], l \in [b]                                                                   \\
              s                           & = ai + j, \hspace{2cm} i \in [b], j \in [a]                                                                   \\
              \intertext{Insert these values to $s, t$:}
              \hat{f}^{b \times a}_{i, j} & = \sum_{l = 0}^{b - 1} \sum_{k = 0}^{a - 1} \omega^{(ai + j)(bk + l)} f_{k, l}^{a \times b}                   \\
                                          & = \sum_{l \in [b]} \sum_{k \in [a]} \omega^{abik + bjk + ail + jl} f_{k, l}^{a \times b}                      \\
                                          & = \sum_{l \in [b]} \sum_{k \in [a]} \omega^{abik} \omega^{bjk} \omega^{ail} \omega^{jl} f_{k, l}^{a \times b}
          \end{align*}
          Since $\omega^{ab} = 1$, so is $\omega^{abik}$. Thus the expression can be reduced to:
          \begin{align*}
              \hat{f}_{i, j}^{b \times a} & = \sum_{l \in [b]} \sum_{k \in [a]} \omega^{bjk}\omega^{ail}\omega^{jl}  f_{k, l}^{a \times b} \\
                                          & = \sum_{l \in [b]} \omega^{ail}\omega^{jl} \sum_{k \in [a]} \omega^{bjk} f_{k, l}^{a \times b}
          \end{align*}

          \newpage

    \item

          \newpage

    \item

          \newpage
\end{enumerate}
\end{document}
