\documentclass[11pt,a4paper]{article}
\usepackage{hyperref}
\usepackage[left=1.5in,right=1.5in,top=1.5in,bottom=1.5in]{geometry}
\usepackage{enumitem}
\usepackage{amssymb,amsmath,mathrsfs,amsthm}
\usepackage{parskip, graphicx, float, wrapfig, subcaption}


\setlength{\parindent}{0pt}
\setlength{\parskip}{3mm}

\pagestyle{empty}

%Title and formatting stuff
\usepackage{titlesec}
\titleformat{\chapter}
{\normalfont\Huge\color{aaltoBlue}}{\thechapter}{20pt}{\filleft\Huge}[\vskip 0.5ex\titlerule]
\usepackage{fancyhdr}
\pagestyle{fancy}
\fancyhead[L]{}
\newcommand{\mymarginpar}[1]{\marginpar{\it \small \raggedleft #1}}
\renewcommand*\thesection{\arabic{section}}

% Miscellaneous LaTeX macros
\newcommand{\heading}[1]{{\Large\color{blue!80!black}\textbf{#1}}\par}
\newcommand{\slemp}[1]{{\color{red}#1}}
\newcommand{\slred}[1]{{\color{red}#1}}
\newcommand{\slbf}[1]{{\color{blue}\textbf{#1}}}
\newcommand{\pair}[2]{\langle {#1}, {#2} \rangle}
\newcommand{\Abar}{\bar{A}}
\newcommand{\ALG}{\text{ALG}}
\newcommand{\approxA}{\alpha_{\cal A}}
\newcommand{\argmin}{\operatornamewithlimits{arg\ min}}
\newcommand{\assign}{\mbox{$\; \leftarrow \;$}}
\newcommand{\bbar}{\bar{b}}
\newcommand{\Bin}{\text{Bin}}
\newcommand{\calA}{{\cal A}}
\newcommand{\calC}{{\cal C}}
\newcommand{\calI}{{\cal I}}
\newcommand{\calM}{{\cal M}}
\newcommand{\calN}{{\cal N}}
\newcommand{\calS}{{\cal S}}
\newcommand{\chat}{\hat{c}}
\newcommand{\chatopt}{\hat{c}^*}
\newcommand{\copt}{c^*}
\newcommand{\Cov}{\text{Cov}}
\newcommand{\DC}{\text{DC}}
\newcommand{\delC}{\partial C}
\newcommand{\delS}{\partial S}
\newcommand{\delT}{\partial T}
\newcommand{\diag}{\textit{diag}}
\newcommand{\dtv}{d_{\text{V}}}
\newcommand{\dom}{\mathcal{D}}
\newcommand{\E}{\text{E}}
\newcommand{\eps}{\varepsilon}
\newcommand{\false}{\text{false}}
\newcommand{\Gbar}{\bar{G}}
\newcommand{\Gnp}{{\cal G}(n,p)}
\newcommand{\Int}{\mathbf{I}}
\newcommand{\lambdamax}{\lambda_{\text{max}}}
\newcommand{\LPI}{$\text{LP}_{\text{I}}$}
\newcommand{\LPII}{$\text{LP}_{\text{II}}$}
\newcommand{\NN}{\mathbb{N}}
\newcommand{\om}{\omega}
\newcommand{\Om}{\Omega}
\newcommand{\OPT}{\text{OPT}}
\newcommand{\OPTf}{\text{OPT}_f}
\newcommand{\phase}{\bar{\sigma}}
\newcommand{\phat}{\hat{p}}
\newcommand{\pt}{p^{(t)}}
\newcommand{\rbot}{r^{\bot}}
\newcommand{\rhat}{\hat{r}}
\newcommand{\RENT}{\text{RENT}}
\newcommand{\RR}{\mathbb{R}}
\newcommand{\RRmn}{\mathbb{R}^{m\times n}}
\newcommand{\RRn}{\mathbb{R}^n}
\newcommand{\RRnn}{\mathbb{R}^{n\times n}}
\newcommand{\sgn}{\text{sgn}}
\newcommand{\stot}{$s$-$t$}
\newcommand{\true}{\text{true}}
\newcommand{\st}{\text{s.t.}}
\newcommand{\vivi}{v_i^Tv_i}
\newcommand{\vivj}{v_i^Tv_j}
\newcommand{\Var}{\text{Var}}
\newcommand{\xbar}{\bar{x}}
\newcommand{\xopt}{x^*}
\newcommand{\ybar}{\bar{y}}
\newcommand{\yopt}{y^*}
\newcommand{\zbar}{\bar{z}}
\newcommand{\zopt}{z^*}
\newcommand{\ZIP}{Z^*_{\text{IP}}}
\newcommand{\ZLP}{Z^*_{\text{LP}}}
\newcommand{\ZZ}{\mathbb{Z}}

\newcommand{\tD}[1]{\textrm{\em\color{aaltoGreen}{#1}}}
\newcommand{\tB}[1]{\text{\color{aaltoBlack}{#1}}}
\newcommand{\Pt}{\ensuremath{\mathrm{P}}}
\newcommand{\NP}{\ensuremath{\mathrm{NP}}}
\newcommand{\bra}{\ensuremath{\langle}}
\newcommand{\ket}{\ensuremath{\rangle}}
\newcommand{\DFT}{\ensuremath{\operatorname{DFT}}}
\newcommand{\rev}{\ensuremath{\operatorname{rev}}}
\newcommand{\quo}{\ensuremath{\operatorname{quo}}}
\newcommand{\rem}{\ensuremath{\operatorname{rem}}}
\newcommand{\take}{\ensuremath{\operatorname{\upharpoonright}}}
\newcommand{\ord}{\ensuremath{\operatorname{ord}}}
\newtheorem{fact}{Fact}
\newcommand\Edit{\mathrm{Edit}}
\newcommand\backtrack{\stackrel{\text{back-track}}{\longrightarrow}}

\newcommand\N{\mathbb{N}} % Sorry did not see \NN etc earlier and now my \N  is everywhere :(
\newcommand\R{\mathbb{R}}
\newcommand\Z{\mathbb{Z}}

%--------- definition environment commands --

\theoremstyle{definition}
\newtheorem{definition}{Definition}[section]
\newtheorem{lemma}{Lemma}[section]
\newtheorem{corollary}{Corollary}[section]
\newtheorem{theorem}{Theorem}[section]
\newtheorem{example}{Example}[section]


%--------- STUFF FOR 2021 ITERATION ---------

\usepackage{bbm}
\usepackage{float}
\usepackage{tikz}
\newcommand{\ex}{\text{E}}
\newcommand{\pr}{\text{P}}
\allowdisplaybreaks

% question title
\newcommand{\qtitle}[1]{{\color{aaltoBlue}\textbf{#1}}}

%% algorithm environment
\usepackage[ruled, vlined]{algorithm2e}
\newcommand\mycommfont[1]{\footnotesize\ttfamily\textcolor{aaltoBlue}{#1}}
\SetCommentSty{mycommfont}
\SetFuncSty{text}
\SetKwProg{Fn}{def}{:}{}
\usepackage{ulem}

\newif\ifanswer
\newif\ifgraded
\newcommand{\InsertAnswer}[1]{\vspace{1mm}{\color{aaltoBlue}\ifanswer \textbf{Solution.}#1\fi}}

%an environment to automatically hide answers. Wrap \InsertAnswer with this
\newenvironment{answer}{}

\newcommand{\createheader}[1]{
{\color{aaltoBlue}{
\LARGE\bf CS-E3190 Principles of Algorithmic Techniques \\ [2mm]
{\it #1 \ifgraded -- Graded Exercise \else -- Tutorial Exercise \fi}

}}}


\newcommand{\rulebox}[1]{
  \ifgraded

    \vspace{-3mm}
    {\color{aaltoBlue} \rule{16cm}{1pt}}

    \centering{
      \vspace{2mm}
      \fbox{\begin{minipage}{14.5cm}
          \vspace{2mm} \small \normalfont
          {Please read the following \textbf{rules} very carefully.
            \begin{itemize}[itemsep=0.5mm, leftmargin=15pt]
              \item Do not consciously search for the solution on the internet.
              \item You are allowed to discuss the problems with your classmates but you should \textbf{write the solutions yourself}.
              \item Be aware that \textbf{if plagiarism is suspected}, you could be asked to have an interview with teaching staff.
              \item The teaching staff can assist with understanding the problem statements, but will \textbf{not be giving any hints} on how to solve the exercises.
              \item The use of generative AI tools of any kind is \textbf{not allowed}.
              \item In order to ease grading, we want the solution of each problem and subproblem to start on a \textbf{new page}. If this requirement is not met, \textbf{points will be deduced}.

            \end{itemize}
          }
          \vspace{0.5mm}
        \end{minipage}}}
  \else
    \vspace{-3mm}
    {\color{aaltoBlue} \rule{16cm}{1pt}}
    \vspace{-2mm}
  \fi
}

%-----------------------------------------------
%-----------------------------------------------

\title{CS-E4500 Advanced Course in Algorithms}
\author{Aleksi Kääriäinen  \\
	Aalto University  \\
	}

\begin{document}

\date{\today}

\maketitle

\newpage

\section*{Week 4 exercices}

\begin{enumerate}
    \item
          Let $p = 113$, $s = 10$, $k = 5$ and the secret $\varphi_0 = 99$. Let the points to be evaluated be $\xi_i =  1, 2, \dots s$.
          Selecting $k - 1$ coefficients for $\varphi_{1}  \dots \varphi_{k - 1}$ independently and uniformly at random yields $\{112, 63, 41, 87\}$.
          Thus, $f = 99 + 112x + 63x^2 + 41x^3 + 87x^4 \in \mathbb{Z}_p$ and the share $i$ is the pair $(\xi_i, f(\xi_i))$:

          \begin{center}
              \begin{tabular}{c|c|c}
                  i & $\xi_i$ & $f(\xi_i)$ \\
                  \hline
                  0 & 1       & 63         \\
                  1 & 2       & 35         \\
                  2 & 3       & 3          \\
                  3 & 4       & 9          \\
                  4 & 5       & 36         \\
                  5 & 6       & 8          \\
                  6 & 7       & 16         \\
                  7 & 8       & 92         \\
                  8 & 9       & 96         \\
                  9 & 10      & 55         \\
              \end{tabular}
          \end{center}

          Now we want to show that the secret $\varphi_0$ is recoverable if a single entity knows $\geq k$ shares $(\xi_i, f(\xi_i))$.
          Consider the first $k$ pairs and let us construct the Vandermonde matrix with the values $\xi_0, \dots \xi_4$.
          \begin{align*}
              V = \begin{bmatrix}
                      1^0 & 1^1 & 1^2 & 1^3 & 1^4 \\
                      2^0 & 2^1 & 2^2 & 2^3 & 2^4 \\
                      3^0 & 3^1 & 3^2 & 3^3 & 3^4 \\
                      4^0 & 4^1 & 4^2 & 4^3 & 4^4 \\
                      5^0 & 5^1 & 5^2 & 5^3 & 5^4 \\
                  \end{bmatrix} = \begin{bmatrix}
                                      1 & 1 & 1  & 1  & 1  \\
                                      1 & 2 & 4  & 8  & 16 \\
                                      1 & 3 & 9  & 27 & 81 \\
                                      1 & 4 & 16 & 64 & 30 \\
                                      1 & 5 & 25 & 12 & 60 \\
                                  \end{bmatrix}
          \end{align*}
          Now, inverting the Vandermonde matrix and multiplying it with a vector consisting of valuations of $f(\xi_i)$ should yield
          the coefficients of $f$:
          \begin{align*}
              V^{-1}x = & \begin{bmatrix}
                              1 & 1 & 1  & 1  & 1  \\
                              1 & 2 & 4  & 8  & 16 \\
                              1 & 3 & 9  & 27 & 81 \\
                              1 & 4 & 16 & 64 & 30 \\
                              1 & 5 & 25 & 12 & 60 \\
                          \end{bmatrix}^{-1} \cdot \begin{bmatrix}
                                                       63 \\ 35 \\ 3 \\ 9 \\ 36
                                                   \end{bmatrix}               \\
              =         & \begin{bmatrix}
                              5   & 103 & 10  & 108 & 1  \\
                              3   & 112 & 37  & 29  & 45 \\
                              83  & 9   & 97  & 12  & 25 \\
                              103 & 21  & 110 & 96  & 9  \\
                              33  & 94  & 85  & 94  & 33
                          \end{bmatrix} \cdot \begin{bmatrix}
                                                  63 \\ 35 \\ 3 \\ 9 \\ 36
                                              \end{bmatrix} = \begin{bmatrix}
                                                                  99 \\ 112 \\ 63 \\ 41 \\ 87
                                                              \end{bmatrix}
          \end{align*}
          Indeed, the result is a vector with the coefficients of $f$ in increasing order ordered by the degree of the term, meaning that the
          secret $\varphi_0$ was recovered successfully.
          \newpage

    \item \begin{enumerate}
              \item Apply the defining quality $a = qb + r$ to $f(x)$, where $b = (x - \xi)$:
                    \begin{align*}
                        f(x)   & = q(x)(x - \xi) + r(x)       \\
                        \intertext{Substitute $x \implies \xi$:}
                        f(\xi) & = q(\xi)(\xi - \xi) + r(\xi) \\
                        f(\xi) & = r(\xi)                     \\
                        \intertext{Remember that $\deg{r} < \deg{b} = 1$. This implies that $\deg{r(x)} = 0$, i.e.
                            $r(x)$ is a constant independent of $x$.}
                        f(\xi) & = r                          \\
                        f(\xi) & = f\rem{(x - \xi)}
                    \end{align*}

              \item Let $a = qb + a\rem{b}$. However, $a = pc + a\rem{c}$ too. Substituting $a$ in the first equation with the second:
                    \begin{align*}
                        qb + a\rem{b}          & = pc + a\rem{c}
                        \intertext{Since $c$ divides $b$, $c$ is a multiple of $b$: $b = x \cdot c$, $x \in R$.}
                        q \cdot (x\cdot c) + a\rem{b} & = pc + a\rem{c}              \\
                        q \cdot (x\cdot c) + a\rem{b} & \equiv pc + a\rem{c} \mod{c} \\
                        a\rem{b}               & \equiv a\rem{c} \mod{c}      \\
                        (a\rem{b})\rem{c}      & = a\rem{c}                   \\
                        a\rem{c}               & = (a\rem{b})\rem{c}
                    \end{align*}
          \end{enumerate}

          \newpage

    \item

          \newpage

    \item

          \newpage
\end{enumerate}
\end{document}
