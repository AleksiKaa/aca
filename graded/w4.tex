\documentclass[11pt,a4paper]{article}
\input{auxiliary.tex}

%-----------------------------------------------
%-----------------------------------------------

\title{CS-E4500 Advanced Course in Algorithms}
\author{Aleksi Kääriäinen  \\
	Aalto University  \\
	}

\begin{document}

\date{\today}

\maketitle

\newpage

\section*{Week 4 exercices}

\begin{enumerate}
    \item
          Let $p = 113$, $s = 10$, $k = 5$ and the secret $\varphi_0 = 99$. Let the points to be evaluated be $\xi_i =  1, 2, \dots s$.
          Selecting $k - 1$ coefficients for $\varphi_{1}  \dots \varphi_{k - 1}$ independently and uniformly at random yields $\{112, 63, 41, 87\}$.
          Thus, $f = 99 + 112x + 63x^2 + 41x^3 + 87x^4 \in \mathbb{Z}_p$ and the share $i$ is the pair $(\xi_i, f(\xi_i))$:

          \begin{center}
              \begin{tabular}{c|c|c}
                  i & $\xi_i$ & $f(\xi_i)$ \\
                  \hline
                  0 & 1       & 63         \\
                  1 & 2       & 35         \\
                  2 & 3       & 3          \\
                  3 & 4       & 9          \\
                  4 & 5       & 36         \\
                  5 & 6       & 8          \\
                  6 & 7       & 16         \\
                  7 & 8       & 92         \\
                  8 & 9       & 96         \\
                  9 & 10      & 55         \\
              \end{tabular}
          \end{center}

          Now we want to show that the secret $\varphi_0$ is recoverable if a single entity knows $\geq k$ shares $(\xi_i, f(\xi_i))$.
          Consider the first $k$ pairs and let us construct the Vandermonde matrix with the values $\xi_0, \dots \xi_4$.
          \begin{align*}
              V = \begin{bmatrix}
                      1^0 & 1^1 & 1^2 & 1^3 & 1^4 \\
                      2^0 & 2^1 & 2^2 & 2^3 & 2^4 \\
                      3^0 & 3^1 & 3^2 & 3^3 & 3^4 \\
                      4^0 & 4^1 & 4^2 & 4^3 & 4^4 \\
                      5^0 & 5^1 & 5^2 & 5^3 & 5^4 \\
                  \end{bmatrix} = \begin{bmatrix}
                                      1 & 1 & 1  & 1  & 1  \\
                                      1 & 2 & 4  & 8  & 16 \\
                                      1 & 3 & 9  & 27 & 81 \\
                                      1 & 4 & 16 & 64 & 30 \\
                                      1 & 5 & 25 & 12 & 60 \\
                                  \end{bmatrix}
          \end{align*}
          Now, inverting the Vandermonde matrix and multiplying it with a vector consisting of valuations of $f(\xi_i)$ should yield
          the coefficients of $f$:
          \begin{align*}
              V^{-1}x = & \begin{bmatrix}
                              1 & 1 & 1  & 1  & 1  \\
                              1 & 2 & 4  & 8  & 16 \\
                              1 & 3 & 9  & 27 & 81 \\
                              1 & 4 & 16 & 64 & 30 \\
                              1 & 5 & 25 & 12 & 60 \\
                          \end{bmatrix}^{-1} \cdot \begin{bmatrix}
                                                       63 \\ 35 \\ 3 \\ 9 \\ 36
                                                   \end{bmatrix}               \\
              =         & \begin{bmatrix}
                              5   & 103 & 10  & 108 & 1  \\
                              3   & 112 & 37  & 29  & 45 \\
                              83  & 9   & 97  & 12  & 25 \\
                              103 & 21  & 110 & 96  & 9  \\
                              33  & 94  & 85  & 94  & 33
                          \end{bmatrix} \cdot \begin{bmatrix}
                                                  63 \\ 35 \\ 3 \\ 9 \\ 36
                                              \end{bmatrix} = \begin{bmatrix}
                                                                  99 \\ 112 \\ 63 \\ 41 \\ 87
                                                              \end{bmatrix}
          \end{align*}
          Indeed, the result is a vector with the coefficients of $f$ in increasing order ordered by the degree of the term, meaning that the
          secret $\varphi_0$ was recovered successfully.
          \newpage

    \item \begin{enumerate}
              \item Apply the defining quality $a = qb + r$ to $f(x)$, where $b = (x - \xi)$:
                    \begin{align*}
                        f(x)   & = q(x)(x - \xi) + r(x)       \\
                        \intertext{Substitute $x \implies \xi$:}
                        f(\xi) & = q(\xi)(\xi - \xi) + r(\xi) \\
                        f(\xi) & = r(\xi)                     \\
                        \intertext{Remember that $\deg{r} < \deg{b} = 1$. This implies that $\deg{r(x)} = 0$, i.e.
                            $r(x)$ is a constant independent of $x$.}
                        f(\xi) & = r                          \\
                        f(\xi) & = f\rem{(x - \xi)}
                    \end{align*}

              \item Let $a = qb + a\rem{b}$. However, $a = pc + a\rem{c}$ too. Substituting $a$ in the first equation with the second:
                    \begin{align*}
                        qb + a\rem{b}                 & = pc + a\rem{c}
                        \intertext{Since $c$ divides $b$, $c$ is a multiple of $b$: $b = x \cdot c$, $x \in R$.}
                        q \cdot (x\cdot c) + a\rem{b} & = pc + a\rem{c}              \\
                        q \cdot (x\cdot c) + a\rem{b} & \equiv pc + a\rem{c} \mod{c} \\
                        a\rem{b}                      & \equiv a\rem{c} \mod{c}      \\
                        (a\rem{b})\rem{c}             & = a\rem{c}                   \\
                        a\rem{c}                      & = (a\rem{b})\rem{c}
                    \end{align*}
          \end{enumerate}

          \newpage

    \item
          Let us first construct a binary tree where all leaves are associated with the polynomial $s_v = x - \xi_v$ and all internal nodes with
          the product of its children $s_u = s_{u0}s_{u1}$. Observe that for all $u$,

          \[s_u = \prod_{v\in \{0,1 \}^k}(x - \xi_v),\]

          where $v$ is the binary string of length $k$ associated with a leaf node, and $u$ is a prefix of $v$. Let us then associate each leaf node with
          another polynomial $l_v = \lambda_v$. Then, for each internal node $u$, assign $l_u = l_{u0}s_{u1} + s_{u0}l_{u1}$. We get a binary tree of the
          following structure:

          \begin{center}
            \includegraphics[scale=0.4]{pics/bintree.jpg}
          \end{center}

          We observe that for every internal node $u$, we have:
          \begin{align*}
            l_u & = \sum \lambda_i \prod_{j \neq i} (x - \xi_j), \\
            \intertext{where $u$ is a prefix of both $i$ and $j$. And for the root node $\eps$:}
            l_\eps & = \sum_{i = 0}^{e - 1} \lambda_i \prod_{_{j \neq i}^{j = 0}}^{e - 1} (x - \xi_j)
          \end{align*}
          The cost of each level in the tree is $O(M(e))$ operations using fast multiplication, and since there are
          $O(\log_2 e)$ levels in the tree, the total cost of operations for the computation of the coefficients is
          $O(M(e)\log_2 e)$.
          \newpage

    \item Following the hint, let us apply $\lambda_v = 1$ for all leaf nodes $v$. Thus we get the coefficients for $f$:
          \begin{align*}
            f & = l_\eps = \sum_{i = 0}^{e - 1} \prod_{_{j \neq i}^{j = 0}}^{e - 1} (x - \xi_j)
            \intertext{At the root level:}
            f & = \prod_{_{j \neq i}^{j = 0}} (x - \xi_j)
            \intertext{Evaluate $f$ at $\xi_i:$}
            f(\xi_i) & = \prod_{_{j \neq i}^{j = 0}} (\xi_i - \xi_j)
          \end{align*}
          For the second pass, let us apply for all $v$, 
          \[ \lambda_i = \eta_i f(\xi_i)^{-1} = \eta_i \prod_{_{j \neq i}^{j = 0}} (\xi_i - \xi_j)^{-1} \]
          Putting all this together, at the root of the tree $\eps$, we have:
          \begin{align*}
            l_\eps & = \sum_{i = 0}^{e - 1} \lambda_i \prod_{^{j = 0}_{j \neq i}}^{e - 1} (x - \xi_j) \\
            l_\eps & = \sum_{i = 0}^{e - 1}\left(\eta_i \prod_{_{j \neq i}^{j = 0}} (\xi_i - \xi_j)^{-1} \right) \prod_{^{j = 0}_{j \neq i}}^{e - 1} (x - \xi_j)
          \end{align*}
          Thus we have constructed the Lagrange interpolation polynomial. Remember the result of problem 3 was that $l$ is computable
          in $O(M(e)\log_2 e)$ operations. We used it twice in the construction of the Lagrange interpolation polynomial, thus the construction
          is also $O(M(e)\log_2 e)$.
          \newpage
\end{enumerate}
\end{document}
